\documentclass{beamer}

\usetheme[numbering=fraction]{metropolis}



%% Some recommended packages.
\usepackage{booktabs}   %% For formal tables:
                        %% http://ctan.org/pkg/booktabs
\usepackage{subcaption} %% For complex figures with subfigures/subcaptions
                        %% http://ctan.org/pkg/subcaption


% AMS packages
\usepackage{amsmath}
\usepackage{amssymb}
\usepackage{amsthm}
\usepackage{mathtools}
\usepackage{mdwlist}
\usepackage{pifont}

% Hyper links
\usepackage{url}
\usepackage{hyperref}
\hypersetup{
   colorlinks,
   citecolor=black,
   filecolor=black,
   linkcolor=black,
   urlcolor=black
}

% Miscellaneous
\usepackage{paralist}
\usepackage{graphicx}
\usepackage{epstopdf}
\usepackage{float}
\usepackage{supertabular}
\usepackage{multirow}


% Code highlighting
\usepackage{listings}
\lstset{%
  basicstyle=\ttfamily\small, % the size of the fonts that are used for the code
  keywordstyle=\sffamily\bfseries,
  captionpos=none,
  columns=flexible,
  lineskip=-1pt,
  keepspaces=true,
  showspaces=false,               % show spaces adding particular underscores
  showstringspaces=false,         % underline spaces within strings
  showtabs=false,                 % show tabs within strings adding particular underscores
  breaklines=true,                % sets automatic line breaking
  breakatwhitespace=true,         % sets if automatic breaks should only happen at whitespace
  escapeinside={(*}{*)},
  sensitive=true
}

\lstdefinestyle{sedel}{
  tabsize=2, % sets default tabsize to 2 spaces
  morekeywords={class, interface, super, type,trait,new,def, defrec, if, then, else, new, inherits, Trait, let, in},
  morecomment=[l]{--},
  morecomment=[l]{//},
  morestring=[b]", % 'b' means inside a string delimiters are escaped by a backslash.
  morestring=[b]',
  literate={->}{{$\rightarrow$}}1 {=>}{{$\Rightarrow$}}1 {/\\}{{$\Lambda$}}1,
}

\lstset{style=sedel}

% Revision tools
\usepackage{xspace}
\usepackage{xcolor}
\usepackage{comment}

\newcommand{\turns}{\vdash}
\newcommand{\oftype}{\!:\!}
\newcommand{\subtype}{<:}
\newcommand{\commonsuper}{\Uparrow}
\newcommand{\commonsub}{\Downarrow}
\newcommand{\disjoint}{*}
\newcommand{\disjointimpl}{*_\textnormal{i}}
\newcommand{\disjointax}{*_\textnormal{ax}}
\newcommand{\subst}[2]{\lbrack #2 := #1 \rbrack~}

\newcommand{\yields}[1]{\highlight{$\; \hookrightarrow #1$}}

\newcommand{\ftv}[1]{\textsf{ftv}(#1)}

\newcommand{\binderspace}{\,}
\newcommand{\appspace}{\;}

\newcommand{\inter}{\&}
\newcommand{\union}{|}
\newcommand{\forr}[2]{\forall #1.\binderspace #2}
\newcommand{\fordis}[3]{\for {(#1 \disjoint #2)} {#3}}
\newcommand{\lam}[2]{\lambda #1.\binderspace #2}
\newcommand{\lamty}[3]{\lam {(#1 \oftype #2)} #3}
\newcommand{\blam}[2]{\Lambda #1.\binderspace #2}
\newcommand{\blamdis}[3]{\blam {(#1 \disjoint #2)} #3}
\newcommand{\mergeop}{,,}
\newcommand{\app}[2]{#1 \; #2}
\newcommand{\tapp}[2]{#1 \appspace #2}
\newcommand{\pair}[2]{(#1, #2)}
\newcommand{\proj}[2]{{\code{proj}}_{#1} #2}
\newcommand{\fst}[1]{\app {\code{fst}} {#1}}
\newcommand{\snd}[1]{\app {\code{snd}} {#1}}
\newcommand{\recordType}[2]{\{ #1 : #2 \}}
\newcommand{\recordCon}[2]{\{ #1 = #2 \}}

\newcommand{\true}{\code{True}}
\newcommand{\tyint}{\code{Int}}
\newcommand{\tybool}{\code{Bool}}
\newcommand{\tychar}{\code{Char}}
\newcommand{\tystring}{\code{String}}

\newcommand{\hltext}[2][gray!40]{\colorbox{#1}{#2}}
\newcommand{\hl}[2][gray!40]{%
  \colorbox{#1}{$\displaystyle#2$}}
\newcommand{\mynotes}[3]{{\color{#2} {\sc #1}: #3}}
\newcommand\bruno[1]{\mynotes{bruno}{red}{#1}}
\newcommand\jeremy[1]{\mynotes{jeremy}{blue}{#1}}
\newcommand\name{{\bf SEDEL}\xspace}
\newcommand{\bname}{$ F_{i} $\xspace}

% Ott includes
\input{sections/miniJS.ott.tex}


\title{\name: Safe and Expressive Delegation-Based Programming}
\subtitle{Joint work with Bruno C. d. S. Oliveira}
\date{\today}
\author{Xuan Bi}
\institute{The University of Hong Kong}


\begin{document}
\maketitle

\begin{frame}
  \frametitle{Outline}
  \tableofcontents
\end{frame}


\section{Introduction}

\begin{frame}{Mainstream OOP languages: the covariant model}

  \begin{itemize}
  \item<1-> Mainstream statically-typed OO language (such as Java, C++,C\#, or
    Scala) all use a similar prorgamming model based on classes (we call it
    the \textit{covariant model}).
    \begin{itemize}
    \item Extensions always produce subtypes.
    \item Inheritance and subtyping go along together.
    \end{itemize}

  \item It has been successfully used for over 50 years, surely demonstrated
    its value in practice.

  \item<2-> As pointed by Cook et al. (1989), there are some situations where the
    covariant model doesn't work quite well, and argued for a more
    \textit{flexible model}.
    \begin{itemize}
    \item Inheritance and subtyping should be decoupled.
    \item Extensions do not always produce subtypes.
    \end{itemize}


  \end{itemize}


\end{frame}




\begin{frame}
  \frametitle{Problems of the covariant model}

  \begin{itemize}
  \item Despite being proposed almost 30 years, Cook et al's paper has had not
    much impact on the design of mainstream OO languages.
    \begin{itemize}
    \item It is not as simple as the covariant one.
    \item Not many compelling applications require such flexible model.
    \end{itemize}

  \item We argue the covariant model is problematic for \textit{extensible
      designs} (such as Extensible Visitors or Object Algebras). There are two
    main issues:
    \begin{itemize}
    \item Visitor/Object-Algebra extensions produce supertypes.
    \item Object Algebra combinators require a very flexible form of dynamic
      inheritance (instead of using low-level (type-unsafe) reflection
      techniques).
    \end{itemize}


  \end{itemize}

\end{frame}


\begin{frame}[fragile]
  \frametitle{Limitations of mainstream OOP}

  Following are limitations of mainstream OOP languages that work against
  modularity and reuse:

  \begin{alertblock}{}
  \begin{itemize}
  \item No retroactive super types \& contravariant subtyping refinement
  \item No inheriting the same class multiple times
  \item No dynamic inheritance
  \end{itemize}


  \end{alertblock}


\end{frame}


\begin{frame}[fragile]
  \frametitle{Retroactive super types \& contravariant refinement}

\begin{columns}[t]
\column{0.5\textwidth}

The covariant model bundles extension with subtypes. There are situations where
extensions are actually supertypes (or no relation at all).

\begin{exampleblock}{}
\begin{lstlisting}
// original class
class B {}
// A extends B but is
// a supertype of B
class A inherits B super B {}
\end{lstlisting}

\end{exampleblock}


\pause

\column{0.5\textwidth}

Most Java-like OOP languages support covariant return type refinement, but no
refinement on method types.

\begin{exampleblock}{}
\begin{lstlisting}[language=java]
interface A {Int m(String s);}
// Invalid: B extends and
// refines the argument type
interface B  extends A {
  Int m(Object s);
}
\end{lstlisting}
\end{exampleblock}



\end{columns}

\pause

% \vskip10pt


Combining retroactive super types and contravariant refinement of argument types
proves to be useful in extensible designs.

\end{frame}


\begin{frame}[fragile]
  \frametitle{No inheriting the same class multiple times}

  \begin{columns}[t]

    \column{0.5\textwidth}


    The code is rejected with the message: ``trait A is inherited twice''. The
    restriction seems to be reasonable.

    \begin{exampleblock}{}
\begin{lstlisting}[language=scala]
trait A {
  def m(x : A) = x
}
trait B extends A with A {}
\end{lstlisting}
    \end{exampleblock}


    \column{0.5\textwidth}

    \pause

    The intention would be that the resulting class would have two overloaded
    methods \lstinline{m}.

    \begin{exampleblock}{}
\begin{lstlisting}[language=scala]
trait C[A] {
  def m(x : A) = x
}
trait D extends
  C[Int] with C[Boolean]
\end{lstlisting}

    \end{exampleblock}

  \end{columns}

  % \vskip15pt

  \pause

  In general, for non-parametric classes/traits the restriction makes sense,
  while for parametric classes there are situations where it would make sense to
  inherit from the same class/trait twice.

\end{frame}


\begin{frame}[fragile]
  \frametitle{No dynamic inheritance}

  In mainstream OOP languages, when both inheritance is used and objects are
  created, the classes involved must be \textit{statically} known.

  \begin{exampleblock}{Intersection types in Scala}
\begin{lstlisting}[language=scala]
trait A
trait B
val newAB : A with B = new A with B
\end{lstlisting}
  \end{exampleblock}


  \pause

  However, in Scala it is not possible to dynamically compose two (statically
  unknown) \emph{objects}.

  \begin{exampleblock}{}

\begin{lstlisting}[language=scala]
def merge[A,B] (x: A) (y: B) : A with B = ...
\end{lstlisting}

  \end{exampleblock}




\end{frame}


\begin{frame}
  \frametitle{Our Solution}

  \begin{itemize}
  \item We separate inheritance from subtyping; these are two different
    mechanisms.
  \item We turn to delegation, and have intersection type and disjoint
    polymorphism. The combination is the so-called \textit{dynamically
      composable traits}.
  \item We present \name: a polymorphic, statically-typed and delegation-based
    OO language.
  \end{itemize}



\end{frame}


% \begin{frame}{Contributions}

%   \begin{itemize}[<+->]

%   \item \textbf{The design of \name:} A polymorphic, statically-typed and
%     delegation-based OO language with a powerful form of multiple inheritance
%     mechanism called \textit{dynamically composable traits}.


%   \item \textbf{Improved variants of extensible designs:} We present improved
%     variants of \emph{Object Algebras} and \emph{Extensible Visitors} in \name.

%   \item \textbf{Elaboration of \name into \bname:} We show that how the high-level
%     OOP abstractions of \name can be translated into a calculus with disjoint
%     intersection types and polymorphism.

%   \item \textbf{Implementation and modularization case study:} A case study on
%     modularizing a small JavaScript-like language.

%   \end{itemize}

% \end{frame}

\section{A Tour of \name}

\begin{frame}[fragile]
  \frametitle{Simple traits}

    \begin{exampleblock}{Specifying traits}

\begin{lstlisting}
type Point = { x : Int, y : Int }
trait point(x : Int, y: Int) { self =>
  def x = x
  def y = y
}
\end{lstlisting}

    \end{exampleblock}

      \begin{itemize}
      \item Traits as templates for creating objects.
      \item The type of \lstinline{self} denotes trait requirements.
      \item Creating object like \lstinline{new[Point] point(3,4)}
      \end{itemize}


\end{frame}


\begin{frame}[fragile]
  \frametitle{Inheriting traits}

  A trait can be extended by inheriting all members of other traits.

  \begin{exampleblock}{}
\begin{lstlisting}
type Point3D = Point & { z : Int }
trait point3D(x : Int, y: Int) inherits point(x,y)
  { self : Point3D => def z = self.x }
\end{lstlisting}

  \end{exampleblock}


    \begin{itemize}
    \item Intersection types model subtyping; inheritance and subtyping are
      separated.
    \item \lstinline{self} acts like an \textit{abstract method}; type
      annotation for \lstinline{self} can be any type.
    \end{itemize}

\end{frame}


\begin{frame}[fragile]
  \frametitle{Detecting and resolving conflicts}

  In case of conflicting methods, traits require conflicts to be resolved at the
  level of the composition by the programmer.

  \begin{alertblock}{Bad trait}
\begin{lstlisting}
trait point2(x : Int, y: Int) inherits
  point(x,y) & point3D(x,y)
\end{lstlisting}

  \end{alertblock}

  \begin{exampleblock}{Good trait}
\begin{lstlisting}
trait point2(x : Int, y: Int) inherits
  point(x,y) \ {x : Int } & point3D(x,y) \ {y : Int }
\end{lstlisting}

  \end{exampleblock}

\end{frame}

\begin{frame}[fragile]
  \frametitle{Dynamic instantiation}

  \begin{columns}[t]

\column{0.5\textwidth}

\begin{exampleblock}{Euclidean norm}

\begin{lstlisting}
trait eNorm { self : Point =>
 def norm (x : Int) (y : Int)
... }
\end{lstlisting}


\end{exampleblock}

\column{0.5\textwidth}

\begin{exampleblock}{Manhattan norm}

\begin{lstlisting}
trait mNorm { self : Point =>
 def norm (x : Int) (y : Int)
...}
\end{lstlisting}

\end{exampleblock}

  \end{columns}

% \vskip10pt


The following function takes a trait, \lstinline{norm}, with unknown
implementation, compose it with the statically known \lstinline{point} trait,
and then instantiate it.

  \begin{block}{}
\begin{lstlisting}
def makePoint (norm: Trait[Point,Norm]) =
  new[Point & Norm] point(3,4) & norm
\end{lstlisting}
  \end{block}

\end{frame}


\begin{frame}[fragile]
  \frametitle{Parametric polymorphism and disjointness constraints}

  The \lstinline{merge} operation can be written as

  \begin{exampleblock}{}
\begin{lstlisting}
def merge A [B * A] (x : A) (y : B) : A & B = x ,, y
\end{lstlisting}

  \end{exampleblock}


  \begin{block}{}
    \begin{itemize}
    \item \name supports parametric polymorphism: \lstinline{A} is a type variable.
    \item The notation \lstinline{B * A} is a disjointness constraint:
      \lstinline{B} can only be instantiated to types that are disjoint with
      \lstinline{A}.
      \item Values of intersection types are directly built using the merge
        construct (denoted as \lstinline{,,}).
    \end{itemize}
  \end{block}

\end{frame}


\section{Extensible Designs in \name}

\section{Case Study}


\section{Elaboration of \name into \bname}


\section{Implementation and Discussion}

\section{Future Work}


\begin{frame}[standout]
  Thank you!

\end{frame}




\end{document}
