\section{Introduction}

Mainstream statically-typed Object-Oriented Programming (OOP) languages (such as Java,
C++ C\# or Scala) all use a similar programming model based on
classes. This programming model has its roots on the
origins of OOP in the 1960's in the Simula~\cite{dahl1967simula} language.
We will refer to this model as the \emph{covariant model} for the
remainder of this paper, because in this model inheritance and
subtyping vary in the same way. More concretely
the following is expected in the covariant model:

\begin{itemize}

\item {\bf Extensions always produce subtypes:} In the standard model, when a 
subclass \emph{extends} a class it automatically becomes a 
\emph{subtype} of the super-class. 

\item{\bf Inheritance and subtyping go along together:}
Class extension does two things at once: it inherits code from the
superclass; and it creates a subtype. 

\end{itemize}

The covariant model has been sucessefully used for over 50 years,
so it clearly has demonstrated its value in practice. 
A large part of the success of this model can probably be attributed 
to its relative simplicity. In particular, programmers do not have to think carefully 
about the difference between subtyping and inheritance (indeed many
programmers confuse the two concepts). \bruno{more?}

Nevertheless the study of the theoretical foundations has
taught us that the story about OOP is not quite so simple. Since the
earliest works on the theory of OOP and subtyping, we have known that 
the covariant view of objects is somewhat simplified. Already in
Cardelli's~\cite{} earliest work on the theoretic foundations of OOP,
we knew that functions do not behave in a strictly covariant way. 
However it was only until \citet{cook1989inheritance}'s famous paper on
``\emph{Inheritance is not Subtyping}'' that the issues were discussed 
in more detail. As Cook et al. argued inheritance and
subtyping are different relations: subtyping being a relation on types 
and inheritance being a relation on objects. In the standard model 
the subtype relation is based on the inheritance hierarchy. This 
works very well if extensions produce 
subtypes. However, as Cook et al.'s work famously demonstrated 
this is not always the case. The essential implication of this is that
the covariant model cannot express well programs where inheritance 
and subtyping do not go along together. Following their observations about 
inheritance and subtyping, Cook et al. suggest a more general and 
flexible programing model with the following properties:

\begin{itemize}

\item {\bf Inheritance and subtyping should be decoupled:} 
That is, there should be different mechanisms for class inheritance 
and class/interface subtyping. 

\item {\bf Extensions do not always produce subtypes:} 
There are cases where classes can inherit from other classes without 
producing subtypes. 

\end{itemize}


\begin{comment}
Cardelli's work on calculi for OOP has shown, for example, that
functions are not strictly covariant.  A function of type $A \to B$ is
a subtype of another function $C \to D$ when $B$ is a subtype of $D$
and $A$ is a \emph{supertype} of $C$. This means, for example that a
function of type $Cat \to Int$ \emph{cannot not be subtype} of a
function with type $Animal \to Int$ (assumming that $Cat$ is a subtype
of $Animal$). In fact, only the opposite can
happen: $Animal \to Int$ can be a subtype of $Cat \to Int$.  This is
at odds with the covariant view. Most mainstream OOP languages, such as Java or C\#, address this
disturbance of the covariant view by making methods \emph{invariant} on 
their argument types. In other words, if a class $A$ with method $m$
extends a class $B$ with method $m$, then $A.m$ can only override 
$B.m$ if the parameters types in both method signatures are \emph{exactly 
the same}. Thus mainstream OOP languages restrict the natural subtyping of
functions. Various other issues related to covariance are known. 
For example...

Cook et al.'s work on ``\emph{Inheritance is not Subtyping}''~\cite{}
is another example of how the theory of  OOP languages contradicts 
the simple covariant model. As Cook et al. argued inheritance and
subtyping are different relations: subtyping being a relation on types 
and inheritance being a relation on objects. In the standard model 
the subtype relation is based on the inheritance hierarchy. This 
works very well if extensions produce 
subtypes. However, as Cook et al.'s work famously demonstrated 
this is not always the case. The essential implication of this is that
the standard model cannot express programs that do not follow the 
covariant view of objects well. Following their observations about 
inheritance and subtyping, Cook et al. suggest a more general and 
flexible programing model with the following properties:

\begin{itemize}

\item {\bf Inheritance and subtyping should be decoupled:} 
That is, there should be different mechanisms for class inheritance 
and class/interface subtyping. 

\item {\bf Extensions do not always produce subtypes:} 
There are cases where classes can inherit from other classes without 
producing subtypes. 

\end{itemize}

\end{comment}

Despite being proposed almost 30 years ago, and one of the most
famous papers in OOP, Cook et al.'s paper has not had much impact 
on the design of mainstream OOP languages (although it has influenced 
the design of several academic languages~\cite{}).
%Certainly this is not because
%researchers or designer of OOP language are unaware of the subtleties 
%of covariance and contravariance. Indeed over the years, and for other
%reasons various features have been added to programming languages to 
We believe that there are two primary reasons for the lack of adoption
of their model.  Firstly, the mental programming model is not
as simple as the covariant model. In Cook's model programmers have to
think more carefully on whether extensions produce subtypes or not,
for example.  Thus, it is crutial for programmers to understand the
difference between subtyping and inheritance.
Secondly, and perhaps more importantly, there are not many compelling applications in
the literature where the need for a more flexible OOP model is
necessary. Thus language designers may argue that the costs outweight 
the benefits, and may decide to stick instead to the covariant 
model, which is simple, and well-understood by programmers. 
%Indeed a famous instance of this is the design of DART

This paper has three primary goals. The first goal is to argue that the
covariant model significantly restricts statically-typed OOP
programming in terms of modularity and reuse for important practical
applications. The second goal to identify additional desirable
features that improve flexibility of OOP and are needed in practice.
In particular we argue that supporting a more \emph{dynamic}
form of inheritance (where concrete implementations of the inherited
code are possibly unknown) is highly desirable in practice. Thus 
we are naturally led to a OO language design using \emph{delegation} (or \emph{dynamic inheritance}).
Finally, we present \name: a programming language that puts these 
ideas into practice using a type system based on 
\emph{disjoint intersection types}~\cite{oliveira2016disjoint} and \emph{disjoint
polymorphism}~\cite{alpuimdisjoint}.

Regarding the first goal, we argue that there are at least two
applications areas where the inflexibility of type systems of
mainstream statically-typed OOP languages is problematic.

\paragraph{\bf JavaScript-style Mixin-Based Programming}
A common programming pattern in JavaScript is based on a variant of
Mixins. This programming style is very flexible and enables forms 
of reuse not usually available in more statically typed languages like Java.
However, mixins in JavaScript fundamentally rely on an \emph{object-level composition}
operator for inheritance~\cite{}. This requires a very dynamic form of
inheritance/delegation that is not available in most class-based 
statically-typed OO languages. Ideally the essence of such 
form of mixins should be capturable in statically-typed languages. 
Languages such as TypeScript do attempt to provide better static
type-checking support for those patterns. However, as recently illustrated 
by the work of \citet{alpuimdisjoint}, there are several issues with such an
approach, including type-unsoundness! 

\paragraph{\bf Extensible Visitors and First-Class Values for Object Algebras}
There has been a remarkable number of works aimed at improving support
for extensibility in programming languages
~\cite{Prehofer97,Tarr99ndegrees,Harrison93subject,McDirmid01Jiazzi,Aracic06CaesarJ,Smaragdakis98mixin,nystrom2006j,togersen:2004,Zenger-Odersky2005,oliveira09modular,oliveira2012extensibility}. 
Some of the more recent work on extensibility is focused on design
patterns such as Extensible
Visitors~\cite{togersen:2004,oliveira09modular} or Object Algebras
\cite{oliveira2012extensibility}. 
Although such design patterns give practical benefits in terms of
extensibility, they also expose limitations in existing mainstream OOP
languages. In particular there are three pressing limitations: 
1) lack of good mechanisms for
  \emph{object-level} composition; 2) \emph{conflation of 
    (type) inheritance with subtyping}; 3) \emph{heavy reliance on generics}.

It turns out that the same composition operator that is required for 
object-level composition in JavaScript, is also needed for the
composition of Object Algebras.
The first limitation shows up, for example, in encodings of Feature-Oriented
Programming~\cite{Prehofer97} or Attribute Grammars~\cite{Knuth1968} using Object
Algebras~\cite{oliveira2013feature,rendel14attributes}. These programs are best
expressed using a form of \emph{type-safe}, \emph{dynamic},
\emph{delegation}-based composition. Although such form of
composition can be encoded in languages like Scala, it requires the
use of low-level reflection techniques, such as dynamic proxies,
reflection or other forms of meta-programming. It is clear
that better language support would be desirable.


Visitors/Object Algebras --> {\bf extensions produce supertypes}
composition requires {\bf dynamic inheritance}. 


\begin{comment}
There has been a remarkable number of works aimed at improving support
for extensibility in programming languages. The motivation behind this
line of work is simple, and it is captured quite elegantly by the
infamous \emph{Expression Problem~}\cite{wadler1998expression}: there
are \emph{two} common and desirable forms of extensibility, but most
mainstream languages can only support one form well. Unfortunately
the lack of support in the other form has significant
consequences in terms of code maintenance and software evolution.  As a
result researchers proposed various approaches to address the problem,
including: visions of new programming
models~\cite{Prehofer97,Tarr99ndegrees,Harrison93subject}; new
programming languages or language
extensions~\cite{McDirmid01Jiazzi,Aracic06CaesarJ,Smaragdakis98mixin,nystrom2006j},
and \emph{design patterns} that can be used with existing mainstream
languages~\cite{togersen:2004,Zenger-Odersky2005,oliveira09modular,oliveira2012extensibility}.

Some of the more recent work on extensibility is focused on design
patterns. Examples include \emph{Object
  Algebras}~\cite{oliveira2012extensibility}, \emph{Modular Visitors}~\cite{oliveira09modular,togersen:2004} or
Torgersen's~\cite{togersen:2004} four design patterns using generics. In those
approaches the idea is to use some advanced (but already available)
features, such as \emph{generics}~\cite{Bracha98making}, in combination with conventional
OOP features to model more extensible designs.  Those designs work in
modern OOP languages such as Java, C\#, or Scala.

Although such design patterns give practical benefits in terms of
extensibility, they also expose limitations in existing mainstream OOP
languages. In particular there are three pressing limitations: 
1) lack of good mechanisms for
  \emph{object-level} composition; 2) \emph{conflation of 
    (type) inheritance with subtyping}; 3) \emph{heavy reliance on generics}.
\end{comment}


The first limitation shows up, for example, in encodings of Feature-Oriented
Programming~\cite{Prehofer97} or Attribute Grammars~\cite{Knuth1968} using Object
Algebras~\cite{oliveira2013feature,rendel14attributes}. These programs are best
expressed using a form of \emph{type-safe}, \emph{dynamic},
\emph{delegation}-based composition. Although such form of
composition can be encoded in languages like Scala, it requires the
use of low-level reflection techniques, such as dynamic proxies,
reflection or other forms of meta-programming. It is clear
that better language support would be desirable.

The second limitation shows up in designs for modelling
modular or extensible visitors~\cite{togersen:2004,oliveira09modular}.  The vast majority of modern
OOP languages combines type inheritance and subtyping. 
That is, a type extension induces a subtype. However
as Cook et al.~\cite{cook1989inheritance} famously argued there are programs where
``\emph{subtyping is not inheritance}''. Interestingly 
not many programs have been previously reported in the literature
where the distinction between subtyping and inheritance is
relevant in practice. However, as shown in this paper, it turns out that this
difference does show up in practice when designing modular
(extensible) visitors.  We believe that modular visitors provide a
compelling example where inheritance and subtyping should
not be conflated!

Finally, the third limitation is prevalent in many extensible
designs~\cite{togersen:2004,Zenger-Odersky2005,oliveira09modular,oliveira2013feature,rendel14attributes}.
Such designs rely on advanced features of generics,
such as \emph{F-bounded polymorphism}~\cite{Canning89f-bounded}, \emph{variance
annotations}~\cite{Igarashi06variant}, \emph{wildcards}~\cite{Torgersen04wildcards} and/or \emph{higher-kinded
types}~\cite{Moors08generics} to achieve type-safety. Sadly, the amount of
type-annotations, combined with the lack of understanding of these
features, usually deters programmers from using such designs.





There are multiple flavours of inheritance. To avoid confusion, since 
the same terminology is often used in the literature to mean different 
things, we use the following 3 terms in this paper:

\begin{itemize}



\item{{\bf Static inheritance:}} Static inheritance refers to what the
  typical model of inheritance in class-based languages. The
  inheritance model is said to be static because when using class
  extension, the extended classes are statically known at compile-type.
Static inheritance is used in languages such as Java, Scala or C\# or
C++\footnote{Note that C++ templates allow the Mixin pattern, which
  enables ...}.

\item{{\bf Mutable Inheritance:}} Prototype-based languages such as 
Javascript or Self allow another model of inheritance, which we call
\emph{mutable inheritance}. In this inheritance model, self references 
can be changed at any point. 

\item{{\bf Dynamic Inheritance:}} Dynamic inheritance is a less well-known 
model which stands in between static and mutable inheritance.
Dynamic inheritance is a model used by some academic delegation 
OOP languages, such as ...  
Unlike the static inheritance model, with dynamic inheritance 
objects can inherits from other objects which are not statically
known. However, unlike mutable inheritance, the self-reference is not 
mutable and cannot be arbitrarely changed at run-time. 

\end{itemize}

In summary the contributions of this paper are:

\begin{itemize}

\item \name: a type-safe and coherent OO language supporting dynamic
inheritance.

\item An illustration of applications that utilize \name's type system 
for improved modularity and extensibility.

\item Thin-layer of sugar on top of disjoint intersection types and
  polymorphism.

\end{itemize}