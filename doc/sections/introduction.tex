\section{Introduction}
\label{sec:intro}

Mainstream statically-typed Object-Oriented Programming (OOP) languages (such as Java,
C++, C\# or Scala) all use a similar programming model based on
classes. This programming model has its roots on the
origins of OOP in the 1960s in the Simula language~\citep{dahl1967simula}.
We will refer to this model as the \emph{covariant model} for the
remainder of this paper, because in this model inheritance and
subtyping vary in the same way. More concretely
the following is expected in the covariant model:

\begin{itemize}

\item {\bf Extensions always produce subtypes:} In the covariant model, when a 
subclass \emph{extends} a class it automatically becomes a 
\emph{subtype} of the superclass.

\item{\bf Inheritance and subtyping go along together:}
Class extension does two things at once: it inherits code from the
superclass; and it creates a subtype. 

\end{itemize}

The covariant model has been successfully used for over 50 years,
so it clearly has demonstrated its value in practice. 
Part of this success can probably be attributed 
to its relative simplicity. In particular, programmers do not have to think carefully 
about the difference between subtyping and inheritance (indeed many
programmers confuse the two concepts). 

Nevertheless the study of the theoretical foundations of OOP has taught us that
the story is not quite so simple. Since the earliest works on the theory of OOP
and subtyping, we have known that the covariant view of objects is somewhat
simplified. Already in \citeauthor{cardelli1984semantics}'s earliest work on
the theoretic foundations of OOP~\citep{cardelli1984semantics}, we knew that functions do not behave in a
strictly covariant way. However it was only until \citeauthor{cook1989inheritance}'s
famous paper~\citeyear{cook1989inheritance} on ``\emph{Inheritance is not Subtyping}'' that the issues were
discussed in more detail. As \citeauthor{cook1989inheritance} argued inheritance and subtyping are
different relations: subtyping being a relation on types and inheritance being a
relation on objects. In the covariant model the subtype relation is based on the
inheritance hierarchy. This works very well if extensions produce subtypes.
However, as \citeauthor{cook1989inheritance}'s work famously demonstrated this is not always the
case. The essential implication of this is that the covariant model cannot
express well programs where inheritance and subtyping do not go along together.
Following their observations about inheritance and subtyping, \citeauthor{cook1989inheritance}
suggest a more general and flexible programming model (which we call the
\emph{flexible model}) with the following properties:

\begin{itemize}

\item {\bf Inheritance and subtyping should be decoupled:} 
That is, there should be different mechanisms for class inheritance 
and class/interface subtyping. 

\item {\bf Extensions do not always produce subtypes:} 
There are cases where classes can inherit from other classes without 
producing subtypes. 

\end{itemize}


\begin{comment}
Cardelli's work on calculi for OOP has shown, for example, that
functions are not strictly covariant.  A function of type $A \to B$ is
a subtype of another function $C \to D$ when $B$ is a subtype of $D$
and $A$ is a \emph{supertype} of $C$. This means, for example that a
function of type $Cat \to Int$ \emph{cannot not be subtype} of a
function with type $Animal \to Int$ (assuming that $Cat$ is a subtype
of $Animal$). In fact, only the opposite can
happen: $Animal \to Int$ can be a subtype of $Cat \to Int$.  This is
at odds with the covariant view. Most mainstream OOP languages, such as Java or C\#, address this
disturbance of the covariant view by making methods \emph{invariant} on 
their argument types. In other words, if a class $A$ with method $m$
extends a class $B$ with method $m$, then $A.m$ can only override 
$B.m$ if the parameters types in both method signatures are \emph{exactly 
the same}. Thus mainstream OOP languages restrict the natural subtyping of
functions. Various other issues related to covariance are known. 
For example...

Cook et al.'s work on ``\emph{Inheritance is not Subtyping}''~\cite{}
is another example of how the theory of  OOP languages contradicts 
the simple covariant model. As Cook et al. argued inheritance and
subtyping are different relations: subtyping being a relation on types 
and inheritance being a relation on objects. In the covariant model 
the subtype relation is based on the inheritance hierarchy. This 
works very well if extensions produce 
subtypes. However, as Cook et al.'s work famously demonstrated 
this is not always the case. The essential implication of this is that
the covariant model cannot express programs that do not follow the 
covariant view of objects well. Following their observations about 
inheritance and subtyping, Cook et al. suggest a more general and 
flexible programing model (which we call the \emph{flexible model}) with the following properties:

\begin{itemize}

\item {\bf Inheritance and subtyping should be decoupled:} 
That is, there should be different mechanisms for class inheritance 
and class/interface subtyping. 

\item {\bf Extensions do not always produce subtypes:} 
There are cases where classes can inherit from other classes without 
producing subtypes. 

\end{itemize}

\end{comment}

Despite being proposed almost 30 years ago, and one of the most famous papers in
OOP, \citeauthor{cook1989inheritance}'s paper has not had much impact on the design of mainstream OOP
languages (although it has influenced the design of several academic
languages~\cite{america1991designing,graver1989type,chambers1992object,bruce1995polytoil}).
%Certainly this is not because
%researchers or designer of OOP language are unaware of the subtleties 
%of covariance and contravariance. Indeed over the years, and for other
%reasons various features have been added to programming languages to 
We believe that there are two primary reasons for the lack of adoption
of their model.  Firstly, the mental programming model is not
as simple as the covariant model. In the flexible model programmers have to
think more carefully on whether extensions produce subtypes or not,
for example.  Thus, it is crucial for programmers to understand the
difference between subtyping and inheritance.
Secondly, and perhaps more importantly, there are not many compelling applications in
the literature where the need for a more flexible OOP model is
necessary. Thus language designers may argue that the costs outweigh
the benefits, and may decide to stick instead to the covariant 
model, which is simple, and well-understood by programmers. 
%Indeed a famous instance of this is the design of DART

This paper has three primary goals. The first goal is to argue that the
covariant model significantly restricts statically-typed OOP in terms of
modularity and reuse for important practical applications. The second goal is to
identify additional desirable features that improve flexibility of OOP and are
needed in practice. In particular we argue that supporting a more \emph{dynamic}
form of inheritance (where concrete implementations of the inherited code are
possibly unknown) is highly desirable in practice. Thus we are naturally led to
a OO language design using \emph{delegation} (or \emph{dynamic
  inheritance})~\cite{ungar1988self,chambers1992object}. Finally, we present
\name\footnote{\name stands for: {\bf S}afe and {\bf E}xpressive {\bf
    DEL}egation}: a purely functional OO language that puts these ideas into
practice using a polymorphic structural type system based on \emph{disjoint
  intersection types}~\cite{oliveira2016disjoint} and \emph{disjoint
  polymorphism}~\cite{alpuimdisjoint}. The choice of structural typing is due to
its simplicity, but we think similar ideas should also work in a nominal type
system.

Regarding the first goal, we show that the inflexibility of type systems of
mainstream statically-typed OOP languages is problematic for \emph{extensible
  designs}. There has been a remarkable number of works aimed at improving
support for extensibility in programming languages
~\cite{Prehofer97,Tarr99ndegrees,Harrison93subject,McDirmid01Jiazzi,Aracic06CaesarJ,Smaragdakis98mixin,nystrom2006j,togersen:2004,Zenger-Odersky2005,oliveira09modular,oliveira2012extensibility}.
Some of the more recent work on extensibility is focused on design patterns such
as Extensible Visitors~\cite{togersen:2004,oliveira09modular} or Object Algebras
\cite{oliveira2012extensibility}. Although such design patterns give practical
benefits in terms of extensibility, they also expose limitations in existing
mainstream OOP languages. In particular there are two main issues:

\begin{enumerate}

\item {\bf Visitor/Object-Algebra extensions produce supertypes:}
  Since in the covariant model extensions always produce subtypes, it
  is very hard to correctly express the subtyping relations between
  Visitor and Object Algebra extensions and the original types.

\item {\bf Object Algebra combinators require a very flexible form of
    dynamic inheritance:} As shown by \citet{oliveira2013feature,rendel14attributes}, Object Algebra
  combinators, which allow a very flexible form of composition for
  Object Algebras, requires features not available in languages like
  Java or Scala.

\end{enumerate}

It is clear that an obvious way to solve the first issue is to move away from
the covariant model, and this is precisely what \name does. For the second
issue, \citet{oliveira2013feature,rendel14attributes} do show how to encode
Object Algebra combinators in Scala. However, this requires the use of low-level
(type-unsafe) reflection techniques, such as dynamic proxies, reflection or
other forms of meta-programming. Better language support is desirable. \name
does this by embracing delegation, and having a powerful polymorphic type system
with intersection types and disjoint polymorphism. \name's type system is
capable of statically type checking the code for delegation-based Object Algebra
combinators and their uses, while at the same time ensuring that the composition
is \emph{conflict-free}. The mechanism that enables this is called
\emph{dynamically composable traits}, since the model is quite close to
traits~\cite{scharli2003traits}, but based on dynamic inheritance.

The theoretical foundations of \name are grounded on the work by
\citet{oliveira2016disjoint} on disjoint intersection types, and the work by
\citet{alpuimdisjoint} on disjoint polymorphism. However
\citeauthor{oliveira2016disjoint} and \citeauthor{alpuimdisjoint}
 do not show how to build high-level OOP constructs for
delegation-based programming, discuss the applications to extensible designs, or
provide a language implementation and conduct case studies to evaluate a
language design. What their work does is to develop core calculi that support an
expressive form of \emph{intersection
  types}~\cite{coppo1981functional,pottinger1980type} with a \emph{merge
  construct}~\cite{dunfield2014elaborating}. In \name subtyping is modelled with
disjoint intersection types, and inheritance is modelled with the merge
construct. \name inherits the key properties of these calculi, which are
type-safety and coherence. The coherence property ensures that the semantics of
\name is unambiguous. In particular this property is useful to ensure that
programs using multiple dynamic inheritance are free of conflicts/ambiguities
(even when the types of the object parts being composed are not statically
know).
  
\begin{comment}
The
novelty of the work in this paper is a three-fold. Firstly we show how
to develop and implement a statically typed, delegation-based OOP
source language on top of core language constructs provided by
disjoint intersection types. Secondly we illustrate the applications
of those high-level constructs to solve issues that show up in
extensibility designs. Finally, we provide a case study on
modularization of language components.
\end{comment}

In summary the contributions of this paper are:

\begin{itemize}

\item {\bf The design of \name}: We present a language design for a polymorphic,
  statically-typed and delegation-based programming language. In this design
  inheritance and subtyping are separated, and a powerful form of conflict-free
  multiple dynamic inheritance is supported.

\item {\bf Improved variants of extensible designs:} We present
  improved variants of \emph{Object
    Algebras} and \emph{Extensible Visitors} in \name. 

\item {\bf A practical example where ``Inheritance is not Subtyping'' matters:}
  This paper shows that extensible Object-Algebras/Visitors suffer from the
  ``inheritance is not subtyping'' problem.

\item {\bf Elaboration of dynamically composable traits into disjoint
    intersection types/polymorphism:} We show that, with a simple syntax directed layer of
  sugar, the high-level OOP abstractions of \name can be translated into a
  calculus with disjoint intersection types and polymorphism. The elaboration is
  based on the work by \citet{cook1989denotational} on the denotational semantics of
  inheritance.

\item {\bf Implementation and modularization case study:} \name is implemented
  and available online. To evaluate \name in terms of its support for more
  modular design we conduct a case study, based on \citeauthor{poplcook}'s undergraduate
  Programming Languages book~\cite{poplcook}, where the features of a small
  subset of a JavaScript-like language are modularized using \name's improved
  support for Object Algebras\footnote{{\bf Note to reviewers:} Due to the
    anonymous submission process, the code for the implementation and case
    studies is submitted as supplementary material.}.

\end{itemize}

\begin{comment}

\paragraph{\bf JavaScript-style Mixin-Based Programming}
A common programming pattern in JavaScript is based on a variant of
Mixins. This programming style is very flexible and enables forms 
of reuse not usually available in more statically typed languages like Java.
However, mixins in JavaScript fundamentally rely on an \emph{object-level composition}
operator for inheritance~\cite{}. This requires a very dynamic form of
inheritance/delegation that is not available in most class-based 
statically-typed OO languages. Ideally the essence of such 
form of mixins should be capturable in statically-typed languages. 
Languages such as TypeScript do attempt to provide better static
type-checking support for those patterns. However, as recently illustrated 
by the work of \citet{alpuimdisjoint}, there are several issues with such an
approach, including type-unsoundness!

\end{comment}