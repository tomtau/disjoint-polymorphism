\section{A Tour of \name}
\label{sec:traits}

This section introduces and showcases the main features of \name. In particular
we focus on \name's native support for \textit{dynamically composable traits}.
Along the way, we also explain briefly various other features of \name.
As a running example, we describe a representation of graphical objects such as
circles, ovals, or buttons. The example is illustrative of typical uses of
JavaScript-style mixins, and it is adapted from an online
tutorial~\footnote{\url{https://javascriptweblog.wordpress.com/2011/05/31/a-fresh-look-at-javascript-mixins/}}.
We will use traits to structure the representation and factor out reusable
components. All code snippets in this and later sections are runnable in our
prototype implementation.


\subsection{Simple Traits}

\name natively supports a simple, yet expressive form of
traits~\cite{scharli2003traits}. Traits provide a mechanism of code reuse in
object-oriented programming, which can be used to model disciplined forms of
multiple inheritance. One interesting aspect about traits is the way conflicting
features that typically arise in multiple inheritance are dealt with. Instead of
automatically resolved by scoping rules, conflicts are, in the case of \name,
detected by the type system, and explicitly resolved by the programmer.
There are three interesting points about \name's traits: 1) they are
\emph{statically typed}; 2) they support \emph{constructors}; and 3)
they support \emph{dynamic inheritance}. The support for such
combination of features is one of the key novelties of \name. This, in
combination with the separation of inheritance and subtyping makes the
trait system particularly flexible and expressive.

In the remainder of this section, we demonstrate various trait features in
\name. A comparison with the traditional trait model can be found in
Section~\ref{sec:related}. The desugaring process of traits is
discussed in Section~\ref{sec:desugar}.

\paragraph{Specifying Traits}
A trait is a collection of related methods that characterize only
a specific perspective of the features of an object. Therefore, compared with
programs using class-based inheritance, programs using traits usually have a large number of
small traits rather than fewer but larger classes. Code reuse with traits is
easier than with classes, since traits are usually shorter and can be
\textit{composed}. % It is the ease of composition that makes traits such a
% appealing language feature: two traits can be freely ``added''
% together.
Trait composition is
a symmetric operation, and trait systems provide conflict detection.

Here is a simple trait \lstinline{point} with its two coordinates \lstinline{x}
and \lstinline{y}\footnote{\name only has floating numbers.}.
\lstinputlisting[linerange=4-7]{../examples/box.txt}% APPLY:linerange=POINT_DEF

%The syntax is very similar to Scala's, with the \lstinline{def} keyword
%beginning a term declaration.

The first thing worth noting is that there are two declarations here.
%At this point the reader may wonder what is the \lstinline{type} declaration
%(\lstinline$Point$) for. 
In mainstream OO languages such as Java, a class declaration such as
\lstinline[language=java]$class C { ... }$ does two things at the same time:

\begin{itemize}
\item Declaring a \textit{template} for creating objects;
\item Declaring a new \textit{type}.
\end{itemize}

In contrast, trait declarations in \name only do the former. Type declarations
are defined separately. For example, the type declaration for \lstinline{Point}
declares the types of the (immutable) fields \lstinline{x} and \lstinline{y}.
Separating the two roles of classes has advantages in terms of flexibility,
although one could argue that it is long-winded sometimes. It is easy enough to
add some syntactic sugar to do the two roles with a single definition, but we
will stick to separate definitions in this paper to illustrate its value.

The trait \lstinline{point} provides a simple example of trait
declarations in \name. The following discussions illustrate 
some fundamental features of \name's traits.

\paragraph{Traits as templates for creating objects} An obvious difference of
traits in \name to many other models of
traits~\cite{scharli2003traits,fisher2004typed,odersky2005scalable} is that they
directly serve as templates for objects. In many other trait models, traits are
complemented by classes, which take the responsability for object creation. In
particular, most models of traits do not allow constructors for traits. As the
\lstinline{point} definition illustrates, however, traits in \name have a single
constructor, which can have an arbitrary number of arguments.

\paragraph{Trait requirements versus provided functionality}
When modelling a trait there are two important aspects: what are the
\emph{requirements} of a trait; and what is the functionality that a trait
\emph{provides}? The requirements of a trait denote the types/methods that the
trait needs to support defining the functionality the trait provides. The trait
\lstinline{point} requires nothing, and provides implementations for the fields
\lstinline{x} and \lstinline{y}. In \name, the type of \lstinline$self$ denotes
what types/methods are required by a trait. \name uses a syntax (inspired by
Scala self type declarations) where the self-reference is explicitly named and
can optionally be given a type. When there is no type annotation on
\lstinline{self}, this means that the trait requires nothing. In this case the
type of \lstinline{self} is the \emph{top} type. This is different from typical
OO languages, where the default type of a class is the same as the class being
defined.

%The purpose of
%declaring types is to use them for type annotations of the self-reference, and
%creating instances. In the trait literature, a trait usually requires a set of
%methods that server as parameters for the provided behaviour. In \name, the type
%of \lstinline$self$ denotes what methods are required.

\paragraph{Creating objects}
We use the \lstinline{new} keyword to create an object. A difference to other OO
languages is that the \lstinline{new} keyword specifies both the intended type
of the object, and how to construct the object using constructors.
\lstinputlisting[linerange=11-11]{../examples/box.txt}% APPLY:linerange=POINT_TEST

\paragraph{Inheriting traits}
A trait can be extended by inheriting all members of
other traits, and defining additional members. This is, in some sense, similar
to class inheritance in traditional OO languages. Traits can inherit
from one or more traits, provided that
no conflicts arise. The trait \lstinline$circle$ extends
\lstinline{point} with an extra field \lstinline{radius}.
\lstinputlisting[linerange=17-19]{../examples/box.txt}% APPLY:linerange=CIRCLE_DEF

The keyword \lstinline{inherits} is one of the two options in \name to introduce
inheritance. In the above example, the trait \lstinline{circle} inherits from
the trait \lstinline{point} two fields \lstinline{x} and \lstinline{y}
and defined its own field \lstinline{radius}.

\paragraph{Intersection types model subtyping}
The type \lstinline{Circle} is defined as the intersection of the type
\lstinline{Point} and a record type with the field \lstinline{radius}. An
intersection type~\cite{coppo1981functional,pottinger1980type} such as
\lstinline{A & B} contains exactly those values which can be used as values of
type \lstinline{A} and of type \lstinline{B}, and as such, \lstinline{A & B}
immediately introduces a subtyping relation between itself and its two
constituent types \lstinline{A} and \lstinline{B}. In the previous example,
\lstinline{Circle} is, unsurprisingly, a subtype of \lstinline{Point}.

\paragraph{Inheritance and subtyping are separated}
Now it is clear that, unlike the common covariant
model, \name separates the concept of subtyping from inheritance. Those two
concepts are not necessarily entangled, although in this particular example,
inheritance goes along with subtyping.

\subsection{Traits with Requirements and ``Abstract'' Methods}

So far the traits we have seen have no requirements. However, very often it is
necessary to call methods using the self-reference. When this happens
we need to express explicit requirements on the traits. An interesting aspect
of \name's trait model is that there is no need for \emph{abstract
  methods}. Instead, abstract methods can be simulated as requirements 
of a trait. 

In our example, each graphical object can be decomposed into two aspects -- its
geometry and its functions. In case of circles, we already have a trait
\lstinline{circle} representing its geometry. We now proceed to define its
functions by another trait (\lstinline{T} is the top type which has a single
value \lstinline{()}).
\lstinputlisting[linerange=27-31]{../examples/box.txt}% APPLY:linerange=CIRCLE_FNS

\noindent Note how in \lstinline$circleFns$ the type of the self-reference is
\lstinline$Circle$, which declares a \lstinline{radius} field that is needed for
the definitions of the methods in the trait. Note also that the trait itself
does not actually contains a \lstinline{radius} definition. In many other OO
models a similar program could be achieved by having an \emph{abstract} radius
definition. In \name there are no abstract definitions (methods or fields), but
a similar result can be achieved via trait requirements.

Requirements are satisfied at object creation time. When \lstinline$circleFns$
is instantiated and composed with other traits, it must be composed with an
implementation of \lstinline$Circle$. For example:
\lstinputlisting[linerange=35-35]{../examples/box.txt}% APPLY:linerange=CIRCLE_FULL

The object \lstinline{circleWithFns} enables us to call
methods from different traits on a single object.
% , such as % \lstinline{circleWithFns.area()} and \lstinline{circleWithFns.radius}

\paragraph{Composition of traits}
The definition of \lstinline{circleWithFns} also shows the second option to
introduce inheritance, namely by \textit{composition} of traits. Composition of
traits is denoted by the operator \lstinline{&}. Thus \name offers two options
when it comes to inheritance: we can either compose beforehand when declaring
traits (using \lstinline{inherits}), or compose at the object creation point
(using \lstinline{new} and the \lstinline{&} operator).

%Under the hood, inheritance is accomplished by using the \textit{merge operator}
%(denoted by \lstinline{,,}). The merge operator~\cite{dunfield2014elaborating}
%allows two arbitrary values to be merged, with the resulting type being an
%intersection type. 
%For example the type of \lstinline{2 ,, true} is
%\lstinline{Int & Bool}.
\begin{comment}
\paragraph{Mutually dependent traits} When two traits are composed, any two
methods in those two traits can refer to each other via the self-reference. We
say these two traits are \textit{mutually dependent}. The next example, though a
bit contrived, illustrates this point.
\lstinputlisting[linerange=5-12]{../examples/evenOdd.txt}% APPLY:linerange=EVEN_ODD

\noindent By utilizing trait requirements, the \lstinline{isEven} and
\lstinline{isOdd} methods can refer to each other in two different traits.
\end{comment}


\paragraph{Multiple trait inheritance} To further demonstrate multiple
trait inheritance, consider adding buttons. The trait of buttons and its type are:
\lstinputlisting[linerange=40-41]{../examples/box.txt}% APPLY:linerange=BUTTON_DEF

Similarly we define functions for buttons as well.
\lstinputlisting[linerange=45-48]{../examples/box.txt}% APPLY:linerange=BUTTON_FNS

\noindent To make a round button from the existing traits the trait
\lstinline{roundButton} just inherits everything from the traits
\lstinline{circle} and \lstinline{button}:
\lstinputlisting[linerange=52-54]{../examples/box.txt}% APPLY:linerange=ROUNDBUTTON_DEF

\noindent Finally we create a round button on the fly and test its functionality
as follows:
\lstinputlisting[linerange=107-110]{../examples/box.txt}% APPLY:linerange=ROUNDBUTTON_TEST

\subsection{Detecting And Resolving Conflicts in Trait Composition}
\label{sec:conflicts}

A common problem in multiple inheritance are conflicts. For example, when
inheriting from two traits that have the same field, then it is unclear which
implementation to choose. There are various approaches to deal with conflicts.
The trait-based approach requires conflicts to be resolved at the level of the
composition by the programmer, otherwise the program is rejected by the type
system.

The following example shows how conflicting methods in two traits are detected
and resolved. Let us introduce an oval shape.
\lstinputlisting[linerange=59-61]{../examples/box.txt}% APPLY:linerange=ASOVAL_DEF

The following trait gets rejected because both \lstinline{asOval} and
\lstinline{circle} have a conflicting \lstinline{radius} field.
\lstinputlisting[linerange=66-68]{../examples/box.txt}% APPLY:linerange=CONFLICT_DEF

\noindent As mentioned in Section~\ref{sec:intro}, \name's type system is based
on intersection types. More concretely, \name uses a type system based on
\emph{disjoint intersection types}~\cite{oliveira2016disjoint}. Disjointness, in
its simplest form, means that the set of values of both types are disjoint. The
above problematic program is ill-typed precisely because both
traits (\lstinline{circle} and \lstinline{asOval}) have the same type for the
\lstinline{radius} field, thus violating the disjointness conditions.

\paragraph{Resolving conflicts}
To resolve the conflict, the programmer needs to explicitly state which
\lstinline{radius} gets to stay. \name provides such a means , the so-called
\textit{exclusion} operator (denoted by \lstinline{\}), which allows one to
avoid a conflict before it occurs. The following is one choice, and is accepted
by \name again.
\lstinputlisting[linerange=74-75]{../examples/box.txt}% APPLY:linerange=CONFLICT_RESOLVE

% \noindent Note that because of the disjointness conditions, the exclusion
% operator is able to pinpoint the exact field given a label and a type.


\subsection{Dynamic Instantiation}

One important difference with traditional traits or classes is that traits in
\name are quite dynamic: we are able to compose traits \textit{dynamically} and
then instantiate them later. This is impossible in traditional OO languages, such
as Java, since classes being instantiated must be known statically. In \name, as
we will explain in Section~\ref{sec:desugar}, traits are just terms: they are
first-class values and can be passed around or returned from other functions.

Let us extend the functions of circles by another method \lstinline{inCircle},
which, given a point, tests if the point lies inside the circle.
\lstinputlisting[linerange=88-90]{../examples/box.txt}% APPLY:linerange=CIRCLE_FNS2

\lstinline{CircleFns2} extends \lstinline{CircleFns} by another method
\lstinline{inCircle}, which, as shown in trait \lstinline{circleFns2}, is
implemented by invoking the \lstinline{norm} method from something of type
\lstinline{Circle & Norm}. So what is this \lstinline{Norm} type? It consists of
a single method \lstinline{norm} -- distance of a point to the origin. We
provide two different norms via two traits.
\lstinputlisting[linerange=79-84]{../examples/box.txt}% APPLY:linerange=NORM_DEF

To facilitate creating round buttons with different norms baked in, we define a
factory that takes a trait \lstinline$norm$ and produces a round button:
\lstinputlisting[linerange=94-96]{../examples/box.txt}% APPLY:linerange=POINT_FUNC

\noindent Here \lstinline{norm}, which is a trait used in the object creation of
\lstinline{roundButtonFac}, is \emph{parameterized}. To express the
type of such unknown trait \name has a special 
\lstinline{Trait} type constructor, and \lstinline$Trait[Point, Norm]$
denotes the type of traits that conforms to the \lstinline$Norm$ type with dependency
on \lstinline{Point}. This gives us the
flexibility to choose different norms at \emph{run-time}. Below is a
round button that provides \lstinline{inCircle} method with Euclidean norm baked
in.
\lstinputlisting[linerange=100-100]{../examples/box.txt}% APPLY:linerange=ROUNDBUTTON_TEST2

This concludes our running example of presenting graphical objects.

\subsection{Parametric Polymorphism and Disjointness Constraints}
\label{sec:merge-construct}

Before we finish this tour, let us revisit the incapability of Scala's
intersection types to define the \lstinline{merge} operation. As pointed out in
Section~\ref{sec:scala-merge}, intersection types in Scala are not completely
first class, since it is hard to build values of intersections with unknown
components. In \name, however, this is trivial. The \lstinline{merge} operation
discussed in Section~\ref{sec:scala-merge} can be written as:
\lstinputlisting[linerange=4-4]{../examples/merge.txt}% APPLY:linerange=MERGE_EG

\noindent Firstly \name supports parametric polymorphism, and type variables are
identifiers beginning with uppercase letters. Secondly, from the typing point of
view, the difference between \lstinline{merge} in Scala and \name is that the
type variable \lstinline{B} now has a \emph{disjointness
  constraint}~\cite{alpuimdisjoint}. The notation \lstinline{B * A} means that
the type variable \lstinline{B} can be instantiated to any types that are
disjoint to the type \lstinline{A}. Finally, the value of the intersection type
\lstinline{A & B} is directly built using the \textit{merge construct} (denoted
by \lstinline{,,}). The merge construct~\cite{dunfield2014elaborating} allows
two arbitrary values (and possible polymorphic) to be merged, with the resulting type
being an intersection type.

Disjoint polymorphism~\cite{alpuimdisjoint} sets apart \name
from other OO languages, and enables extensible designs as is elaborated in
Section~\ref{sec:OA}.
