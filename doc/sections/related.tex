\section{Related Work}
\label{sec:related}

% \bruno{I think (part of) this text can be discussed in here instead:


There are multiple flavours of inheritance. To avoid confusion, since the same
terminology is often used in the literature to mean different things, we use the
following 3 terms when comparing related work with ours.

\begin{itemize}
\item{{\bf Static inheritance:}} Static inheritance refers to what the typical
  model of inheritance in class-based languages. The inheritance model is said
  to be static because when using class extension, the extended classes are
  statically known at compile-time.
\item{{\bf Mutable Inheritance:}} Prototype-based languages allow another model
  of inheritance, which we call \emph{mutable inheritance}. In this inheritance
  model, self-references are mutable and changeable at any point.
\item{{\bf Dynamic Inheritance:}} Dynamic inheritance is a less well-known model
  which stands in between static and mutable inheritance. Unlike the static
  inheritance model, with dynamic inheritance objects can inherit from other
  objects which are not statically known. However, unlike mutable inheritance,
  the self-reference is not mutable and cannot be arbitrarily changed at
  run-time.
\end{itemize}

Figure~\ref{fig:comparision} shows the comparison between \name and various
similar languages that follow \citeauthor{cook1989inheritance}'s ``Inheritance is not
Subtyping'' (i.e. the flexible model), as we will explain below.

\begin{figure}[t]
  \centering
  \begin{small}
  \begin{tabular}{|l||c|c|c|c|}
    \hline
    & \bf{Statically typed} & \bf{Polymorphism} & \bf{Meta-theory} & \bf{Inheritance}  \\
    \hline
    \name & \cmark & \cmark & \cmark & Dynamic \\
    \hline
    \textsc{Self} & \xmark & \xmark & \xmark & Mutable \\
    \hline
    Cecil & \cmark & \cmark & \xmark & Static \\
    \hline
    Cook's Modula-3 & \cmark & \xmark & \xmark & Static \\
    \hline
    IFJ & \cmark & \xmark & \cmark & Dynamic \\
    \hline
  \end{tabular}
  \end{small}
  \caption{Comparison between \name and various similar languages that
  adopt the \emph{flexible model}.}
  \label{fig:comparision}
\end{figure}



% \paragraph{Dynamically-typed Languages with Delegation Mechanism}

% \begin{itemize}
% \item Clojure Protocols
%   % http://www.ibm.com/developerworks/library/j-clojure-protocols/
% \item Ruby mixin
% \item JS mixin
% \end{itemize}

% They are all dynamically typed.


\paragraph{Delegation-based languages}

\citet{lieberman1986using} is the first to promote the use of prototypes and
delegation as the mechanism to code sharing between objects. Since then many
researchers have studied the mechanisms of
delegation~\cite{wegner1987dimensions,malenfant1995semantic,goldberg1989smalltalk}.
\textsc{Self}~\cite{ungar1988self} is a dynamically typed, prototype-based
language with a simple and uniform object model. \textsc{Self}'s inheritance
model is typical of what we call mutable inheritance, because an object's parent
slots may be assigned new values at run-time. Mutable inheritance is rather
unstructured, and oftentimes access to any clashing methods will generate a
``messageAmbiguous'' error at run-time. Although \name's dynamic inheritance is
not as powerful as mutable inheritance, its static type system can guarantee
that no such errors occur at run-time.

There is not much work on statically-typed, delegation-based languages.
\citet{kniesel1999type} provides a good overview of problems when combining
delegation with a static type discipline. Cecil~\cite{chambers1992object,
  chambers1993cecil} is a prototype-based language, where delegation is the
mechanism for method call and code reuse. Cecil supports a polymorphic static
type system, although no meta-theory of any kind is given. Its type system is
able to detect statically when a message might be ambiguously defined as a
result of multiple inheritance or multiple dispatching. However, one major
omission of Cecil, which is also one of the interesting features of \name, is
dynamic inheritance. There are other
works~\cite{fisher1995delegation,anderson2003can} on delegation in a
statically-typed setting, but none of them provide means (such as the merge
construct, disjointness constraints, etc.) that are needed for extensible
designs.

\citet{cook1989inheritance} were the first to propose a typed model of
inheritance where subtyping and inheritance are two separate concepts. In
particular, they introduce the notion of \textit{type inheritance} and show that
inherited objects have inherited types, not subtypes. An interesting aspect of
their calculus is the \textbf{with} construct, used to join two records. This is
somewhat similar to our merge construct. However two major differences are worth
pointing out: 1) the \textbf{with} construct operates only on records; and 2) it
is a biased operator, favoring values from its right argument. This biased
operator is good for modelling mixins, but not traits. The
\textbf{with} construct seems to be unable to merge two arbitrary (and possible
polymorphic) values, since this seems to require something like
\emph{row polymorphism}~\cite{wand1987complete,wand1989type}, which is not available in their language.
The \textit{onion} construct in the Big Bang
language~\cite{palmer2015building,menon2012big} has a similar bias problem -- it is a
left-associative operator which gives rightmost precedence to one
implementation when conflicts exist.

\paragraph{Mixin-based inheritance}

Mixins have become very popular in many OO languages
~\cite{flatt1998classes,bono1999core, ancona2003jam}. \citeauthor{bracha1990mixin}'s
seminal paper~\citep{bracha1990mixin} extends Modula-3 with mixins. Mixins are subclasses parameterized
over a superclass, and used to produce a variety of classes with the same
functionality and behaviour. Mixin-based inheritance requires that mixins be
composed linearly, and as such, conflicts are resolved implicitly (mixins
appearing later overwrite all the identically named features of earlier mixins).
In comparison, the trait model in \name requires conflicts be resolved
explicitly. We want to emphasize that this conflict detection is essential in
expressing composition operators for Object Algebras, without running
into ambiguities.


\paragraph{Trait-based inheritance}

The seminar paper by \citet{scharli2003traits} introduced the ideas behind
traits, where they also documented an implementation of the trait
mechanism in a dynamically typed version of Smalltalk. Since then many
formalizations of traits have been
proposed~\cite{scharli2003traitsformal,ducasse2006traits,bettini2010prototypical}.
For example \citet{fisher2004typed} presented a statically-typed calculus that
models traits. Conflict detection is the hallmark of trait-based
inheritance, compared with mixin-based inheritance. One important difference
with \name is that those systems support \textit{classes} in addition to traits,
and consider the interaction between them, whereas \name is 
delegation based and the mechanism for code reuse is purely traits
(i.e., there are no classes in \name). The
deviation from traditional class-based models is not only because of its
simplicity, but also because we need a very \textit{dynamic} form of
inheritance, as has been elaborated throughout the paper.

Compared to the traditional trait mode, traits in \name have the following
differences: 1) traditional traits cannot be instantiated but only composed with
a class, whereas traits in \name can be instantiated directly; 2) traditional
traits cannot take constructor parameters whereas ours can; 3) the trait system
in \name lacks a proper notation of inheritance relationship. For example in the
traditional trait model, if the same method (i.e., from the same trait) is
obtained more than once via different paths, there is no conflict. This is not
the case in \name; and 4) traits in \name support dynamic
inheritance. 
%In the
%traditional trait model, when it comes to inheritance, the traits being
%inherited must be statically known.




% \citet{flatt1998classes} proposed MIXEDJAVA, an extension to a subset of
% sequential Java called CLASSICJAVA with mixins. In their model, mixins
% completely subsume the role of classes (classes are mixins that do not inherit
% any services). One interesting aspect in their system is that two identically
% named methods are allowed to coexist, and are resolved at run-time with run-time
% context information provided by the current \textit{view} of an object. In
% comparison, conflicts in \name are detected statically, and resolved by the
% programmers. Like \name, their model also enforces the distinction between
% implementation inheritance and subtyping.

% \citet{bono1999core} develop an imperative class-based calculus that provides a
% formal model for both single and mixin inheritance. Objects are represented by
% records and produced by instantiating classes. In their calculus, the class
% construct is extensible but not subtypable, while objects are subtypable but not
% extensible. Like \name, their system has a clean separation between subtyping
% and inheritance. Also, their type system does not have polymorphism.

% \citet{ancona2003jam} extends the Java language to support mixins, called Jam.
% Since Jam is an upward-compatible extension of Java 1.0, it is inheritantly a
% covariant mode. Unlike MIXEDJAVA, mixins can be only instantiated on classes,
% and there is no notion of mixin composition.


\begin{comment}

\begin{itemize}


\item ``Object-Oriented Multi-Methods in Cecil''

\item ``Dimensions of Object-Based Language Design''

\item ``On the Semantic Diversity of Delegation-Based Programming Languages''

\item ``Self: The power of simplicity''

\item ``Type-safe delegation for run-time component adaptation''

\item ``A delegation-based object calculus with subtyping''

\item ``Can Addresses be Types? (a case study: objects with delegation)''

\item ``Inheritance is not subtyping''


Mixins

\item ``mixin-based inheritance''

\item ``Classes and mixins''

\item ``A core calculus of classes and mixins''

\item ``A core calculus of higher-order mixins and classes''

\item ``Jam—Designing a Java Extension with Mixins''



\end{itemize}

Do they have polymorphic type systems? Do they support mutable self reference?

\end{comment}


\paragraph{Class-based languages with more advanced forms of inheritance}

Incomplete Featherweight Java (IFJ), proposed by \citet{bettini2008type}, is a
conservative extension of Featherweight Java with incomplete objects. Besides
standard classes, programmers can also define incomplete classes, whose
instances are incomplete objects. Incomplete objects can be composed (by object
composition) with complete objects, yielding new complete objects at run-time,
while ensuring statically that the composition is type-safe. Incomplete objects
are quite flexible, and support dynamic inheritance. However, object composition
in IFJ is quite restrictive, compared to \name, in that it can only compose an
incomplete object with a complete object. In that regard, and also because IFJ's
type system is not polymorphic, IFJ is unable to encode composition operators of
Object Algebras. \citeauthor{ostermann2002dynamically}'s delegation
layers~\citep{ostermann2002dynamically} use delegation for doing dynamic
composition in a system with virtual classes. This is in contrast with most other
approaches that use class-based composition, but closer to the dynamic
composition that we use in \name.

There are many other class-based OO languages that are equipped with more
advanced forms of
inheritance~\cite{meyer1987eiffel,buchi2000generic,ostermann2001object,kniesel1999type}.
Most of them are heavyweight and are specific to classes. \name is
object-centered, more lightweight, and is dedicated to express extensible
designs in a simpler way.


% Eiffel~\cite{meyer1987eiffel} is a class-based language that is based on the
% identification of classes with types and of inheritance with subtyping. Eiffel
% supports multiple inheritance, with the restriction that name collisions are
% considered programming errors, and ambiguities must be resolved explicitly by
% the programmer (by means of renaming). In this regard, \name is quite like
% Eiffel. However, the type system in \name is more lenient in that two
% identically named methods with different signatures can coexist without any
% problems.

% \citet{kniesel1999type} is the first to show that type-safe integration of
% delegation with subtyping into a class-based model is possible, resulting in the
% DARWIN model. In the DARWIN model, the type of the parent object must be a
% declared class and this limits the flexibility of dynamic composition, whereas
% in \name, the merge operator can merge/compose any objects. Another difference
% with \name lies in the conflict resolution, where DARWIN relies on method
% overriding with the assumption that the author of the overriding method is aware
% of the effect.

% Generic wrappers~\cite{buchi2000generic} supports aggregating objects at
% run-time. In their model, once a ``wrappee'' is assigned to a ``wrapper'', the
% wrappee is fixed. GBETA~\cite{ernst2000gbeta} has some dynamic features that are
% related to delegation. Like Generic wrappers, parents in GBETA are fixed at
% run-time.

% \citet{ostermann2001object} proposed compound references (CR) as a abstraction
% for object references, which provides explicit linguistic support for combining
% different composition properties on-demand. The model is statically typed, and
% decouples subtype declaration from implementation reuse.


% \citet{ostermann2002dynamically} proposed delegation layers as an approach to
% decompose a collaboration into layers and compose these layers dynamically at
% run-time. This combines and generalizes delegation and virtual classes concepts.

% \citet{ostermann2008nominal} compared the nominal and structural subtyping
% mechanisms. They argue nominal subtyping gives more safety guarantee, whereas
% structural subtyping is more flexible from a component-based perspective. The
% type system of \name chooses structural subtyping.

\paragraph{Intersection types, polymorphism and the merge construct}

There is a large body of work on intersection types. Here we only talk about
work that have direct influences on ours. \citet{dunfield2014elaborating} shows
significant expressiveness of type systems with intersection types and a merge
construct. However his calculus lacks coherence. The limitation was addressed
by~\citet{oliveira2016disjoint}, where they introduced the notion of
disjointness to ensure coherence. The combination of intersection types, a merge
construct and parametric polymorphism, while achieving coherence was first
studied in the \bname calculus~\cite{alpuimdisjoint}, where they proposed the
notion of disjoint polymorphism. \bname serves the theoretical foundation of
\name.


\begin{comment}

\begin{itemize}

\item Eiffel

\item ``Delegation by object composition'' (IFJ) and ``Type safe dynamic object
  delegation in class-based languages''

\item ``Dynamically composable collaborations with delegation layers''

\item ``Generic wrappers''

\item ``Object-Oriented Composition Untangled''

\item ``gbeta - a language with virtual attributes, Block Structure, and Propagating, Dynamic Inheritance''

\item ``Nominal and Structural Subtyping in Component-Based Programming''

\item ``Engineering a programming language: The type and class system of Sather ''

\item ``Big Bang Designing a Statically-Typed Scripting Language''

\item ``Building a Typed Scripting Language''



\end{itemize}

\end{comment}
