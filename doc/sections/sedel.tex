\section{A Tour of \name}

This section showcases the novel feature of \name, notably its native support
for dynamically composable traits. We demonstrate by examples that \name
overcomes all those limitations mentioned in Section~\ref{sec:critique}. All
examples shown in this section are runnable in our prototype implementation.


\subsection{Dynamically Composable Traits}
\label{sec:traits}

\name natively supports a simple, yet expressive form of dynamically composable
traits~\cite{scharli2003traits}. Traits provide a mechanism of code reuse in
object-oriented programming, that can be used as an alternative to multiple
inheritance. One interesting aspect about traits is the way conflicting features
that are typical arise in multiple inheritance are dealt with. Instead of
automatically resolved by scoping rules, conflicts are detected by the type
system, and require programmers to explicitly resolve them. In what follows, we
demonstrate various trait features in \name. A comparison with the traditional
trait model can be found in Section~\ref{sec:discuss}. The desugaring process
from traits to \bname is discussed in Section~\ref{sec:desugar}.

\subsubsection{Basic Traits}

A Trait is, in essential, a collection of related methods that characterize only
a specific perspective of the features of an object. Therefore, compared with
programs using inheritance, programs using traits usually have a large number of
small traits rather than fewer but larger classes. Code reuse with traits is
easier than with classes, since traits are usually shorter and can be
\textit{composed}. It is the ease of composition that makes traits such a
appealing language feature: two traits can be freely ``added'' together (which
is an symmetric operation); and trait systems provide conflict detection and so
on.

The first example shows basic trait composition. Many social networking sites
allow users to ``upvote'' a comment and the number of upvotes that comment has
received is also displayed. We would like to separate the logic for upvotes from
comments so that it can be reused in other entities. The code below defines a
trait, \lstinline$comment$, which contains a single method \lstinline$content$.

\lstinputlisting[linerange=4-7]{../examples/comment.txt}% APPLY:linerange=COMMENT_DEF

The syntax is very similar to Scala's, with the \lstinline{def} keyword
beginning a term declaration. Next we create another trait, \lstinline$up$, for
tracking the number of upvotes.

\lstinputlisting[linerange=12-15]{../examples/comment.txt}% APPLY:linerange=UP_DEF

At this point the reader may wonder what are those \lstinline{type} declarations
(\lstinline$Comment$ and \lstinline$Up$) for. In mainstream OO languages such as
Java, a class declaration such as \lstinline[language=java]$class C { ... }$ does two things at
the same time:

\begin{itemize}
\item Declaring a \textit{template} for creating object;
\item Declaring a new \textit{type}.
\end{itemize}

In contrast, trait declarations in \name only do the former. Back to our
example, the purpose of declaring two types is just to use them for type
annotations of the self-reference. In the trait literature, a trait usually
requires a set of methods that server as parameters for the provided behaviour.
In our example, the type of \lstinline$self$ denotes what methods are required.

Creating an object is via the \lstinline{new} keyword, a convention in many OO
languages, except for one crucial novelty: we can create an object from multiple
traits. More precisely, the object is created from the \textit{composition} of
those traits. Therefore, we are able to call methods from different traits on a
single object. For example, we can create an object from \lstinline$comment$ and
\lstinline$up$ and test its functionality as follows:

\lstinputlisting[linerange=20-22]{../examples/comment.txt}% APPLY:linerange=COMP_TEST

\subsubsection{Traits with Dependencies}

The following example shows that a trait can depend on another trait. We define
the type of a point and a trait for a standard point.

\lstinputlisting[linerange=5-9]{../examples/point.txt}% APPLY:linerange=POINT_DEF

The norm of a point can be defined as its distance to the origin. We provide two
definitions of norm via two traits\footnote{Due to the lack of full support for
  arithmetic operations, the definitions of Euclidean and Manhattan norms are
  ridiculously simplified. }.

\lstinputlisting[linerange=16-22]{../examples/point.txt}% APPLY:linerange=NORM_DEF

Note how in \lstinline$euclideanNorm$ and \lstinline$manhattanNorm$ the type of
the self-reference is \lstinline$Point$! This is in contrast to a typical
object-oriented language, such as Java, where the self-reference must always be
of the same type as the class being defined. It is this functionality that
allows us to express dependencies between traits: when \lstinline$euclideanNorm$
and \lstinline$manhattanNorm$ are instantiated and composed with some other
traits, they must be composed with an implementation of \lstinline$Point$.
\lstinputlisting[linerange=31-33]{../examples/point.txt}% APPLY:linerange=NORM_TEST

\subsubsection{Traits and Inheritance}

Traits in \name are extensible. A trait can be extended by inheriting all
members of other traits, and defining additional members. This is, in some
sense, similar to class inheritance in traditional OO languages. However, one
difference is that, a trait can inherits one or more traits without any
problems, provided that there are no conflicts arising.

Imagine we want to describe birds that can both fly and swim. For better reuse,
we define a trait that has the ability to swim:
\lstinputlisting[linerange=5-8]{../examples/birds.txt}% APPLY:linerange=SWIM_DEF
Similarly, here is a trait that has the ability to fly:
\lstinputlisting[linerange=13-16]{../examples/birds.txt}% APPLY:linerange=FLY_DEF
Now we are ready to describe a bird (duck) that can both fly and swim:
\lstinputlisting[linerange=20-23]{../examples/birds.txt}% APPLY:linerange=BIRD_DEF
Notice how trait \lstinline{duck} inherits the \lstinline{swim} and
\lstinline{fly} methods from trait \lstinline{swimming} and \lstinline{flying},
respectively. Here is how this super duck shows off it can both fly and swim:
\lstinputlisting[linerange=27-29]{../examples/birds.txt}% APPLY:linerange=BIRD_TEST


\subsubsection{Detecting Conflicts in Trait Composition}

A conflict arises if and only if we combine two traits that have identically
named methods. The traditional approach to resolve conflicts is by method
overriding, automatically done by the type system, whereas the trait-based
approach enforces conflicts be resolved on the level of the composition by the
programmers, otherwise the program is rejected by the type system. The following
example, adapted from~\citet{alpuimdisjoint}, shows how conflicting methods in
two traits are detected.

\lstinputlisting[linerange=6-16]{../examples/log.txt}% APPLY:linerange=CONFLICT_DEF

Here two traits \lstinline{person} and \lstinline{consoleLogger} are composed in
a variable \lstinline{jim}. It is type-safe to access both methods from
\lstinline{Person} and \lstinline{Loggable} in the object \lstinline{jim}.

If we have another trait \lstinline{employee}, and we mistakenly compose this
with \lstinline$person$, then the program is not accepted, due to both
\lstinline{Person} and \lstinline{Employee} sharing the same name (\lstinline{name})
and the same type (\lstinline$String$).

\lstinputlisting[linerange=21-27]{../examples/log.txt}% APPLY:linerange=CONFLICT_WRONG

\jeremy{We lack proper support for resolving conflicts}


\subsection{Dynamic Instantiation}

One difference with traditional traits or classes is that traits in \name are
quite dynamic: we are able to compose traits \textit{dynamically} and then
instantiate them later. This is impossible in traditional OO language, such as
Java, since classes being instantiated must be known statically. In \name, as we
will explain in Section~\ref{sec:desugar}, traits are just terms: they are
first-class values and can be passed around or returned from a function. For
instance in the point example, we can define a function that takes a trait
\lstinline$norm$, compose it with \lstinline{point} and then instantiate it.
\lstinputlisting[linerange=27-27]{../examples/point.txt}% APPLY:linerange=POINT_FUNC

Here \lstinline{Trait} is a built-in keyword, and \lstinline$Trait[Point, Norm]$
is the type of traits that conform to the \lstinline$Norm$ type with dependency
on \lstinline$Point$. We can then create a point object with the Euclidean norm
as simply as \lstinline$makePoint(euclideanNorm())$.




\subsection{Extensibility and Subtyping}
\label{sec:extensibility}

Various solutions to the Expression Problem~\cite{wadler1998expression} in the
literature~\cite{finally-tagless,oliveira09modular,DelawareOS13,oliveira2012extensibility,
  swierstra:la-carte} are closely related to type-theoretic encodings of
datatypes. Indeed, variants of the same idea keep appearing in different
programming languages, because the encoding of the idea needs to exploit the
particular features of the programming language (or theorem prover).
Unfortunately language-specific constructs obscure the key ideas behind those
solutions. In this subsection, we presents \name's solution to the Expression
Problem that intends to capture the key ideas of various solutions in the
literature.

To start, we begin with a very simple system modeling arithmetic expressions and
evaluation. The initial system constitutes expressions with two variants
(literals and addition), as depicted below:
\lstinputlisting[linerange=4-7]{../examples/visitor.txt}% APPLY:linerange=ALGEBRA_DEF

The first part, captured by the type \lstinline{ExpAlg[E]}, constitutes the
so-called algebra of the datatype. The first component abstracts over the type
of literal expressions (\lstinline{Int -> E}). The second component abstracts
over the type of addition expression (\lstinline{E -> E -> E}).

The second part, which is the actual type of the arithmetic expression, is:
\lstinputlisting[linerange=11-11]{../examples/visitor.txt}% APPLY:linerange=CHURCH_DEF
An attentive reader may immediately recognize this is exactly the visitor
pattern: the method \lstinline{accept} takes an internal visitor as argument.

% \paragraph{Data constructors.} Using \lstinline{Exp} the two data constrcutors
% are defined as follows:
% \lstinputlisting[linerange=52-61]{../examples/visitor.txt}% APPLY:linerange=DC_DEF
% Note that the notation \lstinline{/\E} is type abstraction: it introduces a type
% variable in the environment.


\paragraph{Operations.} Defining operations over expressions requires
implementing \lstinline{ExpAlg[E]}. An interesting operation over expressions is
evaluation. The first step is to choose a suitable concrete type for
instantiating the type parameter \lstinline{E}. One such suitable type is:
\lstinputlisting[linerange=15-15]{../examples/visitor.txt}% APPLY:linerange=EVAL_DEF
Using \lstinline{IEval}, we can define a trait that implements the evaluation
rule for each variant:
\lstinputlisting[linerange=19-22]{../examples/visitor.txt}% APPLY:linerange=EVAL_IMPL


\paragraph{Add a subtraction variant.} Interesting things happen when a new
variant, such as subtraction is added. To do so, we need to extend both
\lstinline{ExpAlg[E]} and \lstinline{evalAlg}:
\lstinputlisting[linerange=27-31]{../examples/visitor.txt}% APPLY:linerange=SUB_DEF
Firstly, \lstinline{SubExpAlg[E]} defines an extended algebra that contains the
variants of the original plus the new subtraction variant. Intersection types
are used to model \textit{type inheritance}. Secondly, a new type of expressions
(\lstinline{ExtExp}) with subtraction is needed. It is important to note that
the \lstinline{accept} method now takes the new algebra \lstinline{SubExpAlg[E]}
as argument.

\paragraph{Inheritance is not subtyping.} In the presence of subtyping, there
are interesting subtyping relations between datatypes and their
extensions~\cite{oliveira09modular}. The interesting aspect about subtyping in
the new system is that subtyping follows the opposite direction of the
extension. In other words, subtyping is contravariant with respect to the
extension. First note that \lstinline{SubExpAlg[E]} appears in parameter
position of \lstinline{accept} and function parameters are natually
contravariant: \lstinline{ExtExp} should be a \textit{supertype} of
\lstinline{Exp}. By contrast, in common OO languages, type inheritance and
subtyping always go along together. For example, Scala forbids any kind of
type-refinement on method parameter types. The consequence of this is that in
those languages, it is simply impossible to express that \lstinline{ExtExp} is
both an extension and a supertype of \lstinline{Exp}. Such kind of phenomena is
an example where ``inheritance is not subtyping''~\cite{cook1989inheritance}.

\paragraph{Add a new operation.} The second type of extension is adding a new
operation, such as pretty printing. Similar to evaluation, the interface of the
pretty printing feature is modeled as:
\lstinputlisting[linerange=37-37]{../examples/visitor.txt}% APPLY:linerange=PRINT_DEF
The implementation is spelled out straightforwardly:
\lstinputlisting[linerange=42-46]{../examples/visitor.txt}% APPLY:linerange=PRINT_IMPL


\paragraph{Usage.} The last piece is to define convenient data constructors:
\lstinputlisting[linerange=52-61]{../examples/visitor.txt}% APPLY:linerange=DC_DEF
So now we can test the shiny new system:
\lstinputlisting[linerange=66-70]{../examples/visitor.txt}% APPLY:linerange=VISITOR_EG
Notice that we are able to pass \lstinline{lit 5}, which is of type
\lstinline{Exp}, to \lstinline{sub}, which expects a value of type
\lstinline{ExtExp}. This works precisely because \lstinline{Exp} is a
\textit{subtype} of \lstinline{ExtExp}.



\subsection{How much dynamic is too much}
\label{sec:dynamic}


\jeremy{ A composition operator combining two algebras at run-time,
  demonstrating the dynamic nature of trait inheritance, also show we can do
  multiple inheritance of things with the same type}

The solution to the Expression Problem presented in
Section~\ref{sec:extensibility} has one wrinkle in its usage. To be able to call
different operations from the same expression \lstinline{e}, we need to make two
copies \lstinline{o1} and \lstinline{o2}: one is used for printing, while the
other is used for evaluation. An ideal approach would be allowing a single object
to be created that supports both the printing and evaluation features.
Unfortunately, dynamic composition of algebras is non-trivial.
\citet{oliveira2012extensibility} addressed this problem by proposing
\textit{object algebra combinators} that combine multiple algebras into one.
However, as they noted, such combinators written in Java are difficult to use in
practice. A version of this combinator has been encoded in Scala using
intersection types and an encoding of the merge
operator~\cite{oliveira2013feature, rendel14attributes}. Again, the Scala
encoding of the merge operator is quite complex as it relies on low-level
type-unsafe programming features such as dynamic proxies, reflection or other
meta-programming techniques. In \name there is no need for such complex
encoding, as the merge operator is natively supported.

The combinator is defined by the \lstinline{combine} trait, which takes two
algebras to create a combined algebra. It does so by appropriately delegating
behaviours in each component algebra to the combined algebra.

\lstinputlisting[linerange=49-54]{../examples/visitor3.txt}% APPLY:linerange=COMBINE1_DEF

Note how \lstinline{combine} requires all the interesting features of \name.
Parametric polymorphism is need because \lstinline{combine} must compose algebras
with arbitrary type parameters. Intersection types are needed because the
resulting algebra will have an intersection type. The merge operator is needed
to compose the input algebras dynamically. Finally, a disjointness constraint is
needed to ensure that two input algebras build values of disjoint types
(otherwise ambiguity could arise).

With \lstinline{combine} at hand, we can finally have a object \lstinline{o}
print and evaluate at the same time:

\lstinputlisting[linerange=72-75]{../examples/visitor3.txt}% APPLY:linerange=COMBINE1_TEST

An acute reader may have smelled some boilerplate code in the definition of
\lstinline{combine}. After all, \lstinline{combine} just invokes the
corresponding method on the arguments, which is solely driven by the types of
the two input algebras and the resulting algebra. It turns out that the type
system behind \name is powerful enough to derive such definition for us:
\lstinline{combine} is no more than just the merge of two algebras!

\lstinputlisting[linerange=60-60]{../examples/visitor3.txt}% APPLY:linerange=COMBINE_DEF

That's it. No more boilerplate necessary. Two points worth noting: 1) \name
supports \textit{dynamic inheritance}. Notice how we let \lstinline{combine1}
``inherits'' \lstinline{f & g}, for which the implementation is unknown
statically. 2) \name supports multiple inheritance of traits with the same type:
both \lstinline{f} and \lstinline{g} are traits of the subtraction algebra.
Mainstream OO languages cannot even imagine being such dynamic and flexible.
\name hits a sweet spot in the design space of being how much dynamic, where we
retain as much flexibility as in a dynamic language without sacrificing the
type-safety guarantee we always love from a static type system.
