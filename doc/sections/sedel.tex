\section{Object Algebras and Extensible Visitors in \name}
\label{sec:OA}

This section shows that \name's features are enough for encoding extensible
designs for Object Algebras and Extensible Visitors that have been presented in
mainstream languages. Moreover \name addresses limitations of those languages
mentioned in Section~\ref{sec:critique}, making those designs significantly
simpler and convenient to use. There are two main advantages of \name over
existing languages:
\begin{enumerate}
\item \name does not couple inheritance and subtyping. Moreover
\name supports contravariant parameter types in the subtyping
relation. This enables \name to account for the \emph{correct
subtyping relations} between datatypes modelled with Object Algebras
and Extensible Visitors.
\item \name supports an expressive form of dynamic inheritance. This form
  of inheritance is sufficient to directly express (delegation-based) composition
  operators for Object Algebras~\cite{oliveira2012extensibility}, without any boilerplate or meta-programming.
\end{enumerate}

These two features avoid the use of low-level (and type-unsafe) programming
techniques, and make the designs less reliant on advanced
features of generics.


\subsection{First-Class Object Algebras Values: A Case for Inheritance
is not Subtyping}
\label{sec:objectalgebra}

\citet{oliveira2012extensibility} proposed a design pattern that can solve the
Expression Problem~\cite{wadler1998expression} in languages like Java. An
advantage of the pattern over previous solutions is that it is relatively
lightweight in terms of type system features. However, a problem with Object
Algebras is that it is hard to pass values built with Object Algebras around,
since there are no concrete expressions/AST types. This problem happens, not
because this is a fundamental limitation of Object Algebras, but because the
type systems of OO languages like Java or Scala are too restricted to express
the types of first-class Object Algebra values directly. Fortunately \name has no
such restriction as illustrated next.

\paragraph{Simple expressions}
To start, we begin with a very simple system modeling arithmetic expressions and
evaluation. The initial system constitutes expressions with two variants
(literals and additions), as depicted below:
\lstinputlisting[linerange=4-4]{../examples/visitor.txt}% APPLY:linerange=ALGEBRA_DEF

\noindent The type \lstinline{ExpAlg[E]}, constitutes the so-called Object
Algebra interface. A concrete Object Algebra will implement such an interface by
instantiating the type parameter with a type that constructs objects for a
particular purpose.

% The first component abstracts over the type of literal
% expressions (\lstinline{Int -> E}). The second component abstracts over the type
% of addition expression (\lstinline{E -> E -> E}).

\paragraph{Operations.} An interesting operation over expressions is
evaluation. The first step is to choose a suitable concrete type for
instantiating the type parameter \lstinline{E}. One such suitable type is:
\lstinputlisting[linerange=26-26]{../examples/visitor.txt}% APPLY:linerange=EVAL_DEF
Using \lstinline{IEval}, we can define a trait that implements the evaluation
rule for each variant:
\lstinputlisting[linerange=30-32]{../examples/visitor.txt}% APPLY:linerange=EVAL_IMPL

\paragraph{Second-class Object Algebra Values} In the original pattern
proposed by \citet{oliveira2012extensibility}, concrete values are built with
functions that abstract over concrete Object Algebras. For example:

\lstinputlisting[linerange=8-8]{../examples/visitor.txt}% APPLY:linerange=MKVALUE_DEF

\noindent builds an expression that adds $2$ and $3$. One way to
interpret this code is that Object Algebras act as \emph{factories}
that build values of some \emph{abstract} type \lstinline{E}. However
suppose we wanted to express something like:

\begin{lstlisting}
def eval (e : Exp) : Double = ...
\end{lstlisting}

\noindent where \lstinline{Exp} here is supposed to represent a
concrete type for values built using Object Algebras. Unfortunately, in languages
such as Java, it is hard to define such concrete type for values
without giving up extensibility and the expected subtyping  relations.

\paragraph{First-class Object Algebra values in \name}
In \name the actual type of arithmetic expressions is:
\lstinputlisting[linerange=12-12]{../examples/visitor.txt}% APPLY:linerange=CHURCH_DEF
An attentive reader may immediately recognize this is a variant of the \textsc{Visitor}
pattern. Indeed, as already noted by \citeauthor{oliveira2012extensibility}, Object
Algebras are closely related to internal visitors~\cite{Oliveira_2008}. However, \emph{Object
Algebras in languages like Java or Scala do not define the type
\lstinline{Exp} because this would preclude extensibility in those
languages}. Nevertheless in \name it is still possible to define
\lstinline{Exp} and retain extensibility. In essence, the type
\lstinline{Exp} simply encapsulates the type
of the \lstinline{mkExp} method. With \lstinline{Exp}, the expression \lstinline{2 + 3} is
built as:

\lstinputlisting[linerange=66-66]{../examples/visitor.txt}% APPLY:linerange=EXPRESSION1_EG

\noindent \name provides a lightweight syntax for writing first-class functions,
e.g., \lstinline{\x -> x + 1}, and big lambda (\lstinline{/\}) to introduce type
variables. Note that the code is almost the same as before, but the expression
now has a concrete type, rather than being a function. With \lstinline{Exp} we
can easily abstract over values, as we will see later.


\paragraph{Add a subtraction variant.} Interesting things happen when a new
variant, such as subtraction is added. To do so, we need to extend both
\lstinline{ExpAlg[E]} , \lstinline{evalAlg} and create a more refined type for
expressions with subtraction:

\lstinputlisting[linerange=37-40]{../examples/visitor.txt}% APPLY:linerange=SUB_DEF

\noindent Firstly, \lstinline{SubExpAlg[E]} defines an extended algebra interface that
contains the variants of the original plus the new subtraction variant.
Intersection types are used to model subtyping. Secondly, \lstinline{subEvalAlg}
inherits from \lstinline{evalAlg} and complements it with the subtraction case.
Finally, a new type of expressions (\lstinline{ExtExp}) with subtraction is
needed. It is important to note that the \lstinline{accept} method now takes the
new algebra \lstinline{SubExpAlg[E]} as argument.

\paragraph{Inheritance is not subtyping.} In the presence of subtyping, there
are interesting subtyping relations between ASTs and their
extensions~\cite{oliveira09modular}. Interestingly in the new system, subtyping
follows the opposite direction of the extension. In other words, subtyping is
contravariant with respect to the extension. First note that
\lstinline{SubExpAlg[E]} appears in a parameter position of \lstinline{accept}
and the function constructor is \textit{contravariant} in parameter types. Thus
\lstinline{ExtExp} should be a \textit{supertype} of \lstinline{Exp}. By
contrast, in common OO languages, subtyping and inheritance always go along
together. For example, Java/Scala forbid any kind of type-refinement on method
parameter types. The consequence of this is that in those languages, it is
impossible to express that \lstinline{ExtExp} is both an extension and a
supertype of \lstinline{Exp}. In \name, however, \lstinline{ExtExp} is indeed a supertype
of \lstinline{Exp}.

\paragraph{Add a new operation.} The second type of extension is adding a new
operation, such as pretty printing. Similar to evaluation, the interface of the
pretty printing feature is modelled as:
\lstinputlisting[linerange=46-46]{../examples/visitor.txt}% APPLY:linerange=PRINT_DEF
The implementation is straightforward:
\lstinputlisting[linerange=51-54]{../examples/visitor.txt}% APPLY:linerange=PRINT_IMPL

\paragraph{Abstracting over Object Algebra values.} We now can abstract over
values of \lstinline{Exp} and \lstinline{ExtExp}. For example the evaluator and
pretty-printer for expressions with subtraction (\lstinline{ExtExp}) are defined
as follows:
\lstinputlisting[linerange=61-62]{../examples/visitor.txt}% APPLY:linerange=EVAL_PRINT
To test our functions we create some values of type \lstinline{ExtExp} as
follows:
%We define two expressions, one of type \lstinline{Exp}, the other
%\lstinline{ExtExp}:
\lstinputlisting[linerange=69-70]{../examples/visitor.txt}% APPLY:linerange=EXPRESSION2_EG
Notice how we reuse \lstinline{e1} in the definition of
\lstinline{e2}, even though they have \emph{different types}. This is
possible by calling the \lstinline{accept}
method of \lstinline{Exp} on a subtype (\lstinline{SubExpAlg})
of \lstinline{ExpAlg}.

\paragraph{Client code}
The following example illustrates a simple test program of the
functionality, using \lstinline{print} and \lstinline{eval}, as well
as the values defined previously:
\lstinputlisting[linerange=80-81]{../examples/visitor.txt}% APPLY:linerange=VISITOR_EG

\noindent Note how \lstinline{print} and \lstinline{eval} can be
safely applied to \lstinline{e1} (of type \lstinline{Exp}) since
\lstinline{Exp} is a \emph{subtype} of \lstinline{ExtExp}.

\subsection{Native Dynamic Object Algebra Composition Support}
\label{sec:dynamic}

When programming with Object Algebras, oftentimes it is
necessary to pack multiple operations in the same object. For example,
in the simple language we have been developing it may be useful to
create an object that contains both printing and evaluation.
%The solution to the Expression Problem presented so far has a wrinkle in its usage. To be able to call
%different operations from the same expression \lstinline{e}, we need to make two
%copies \lstinline{o1} and \lstinline{o2}: one is used for printing, while the
%other is used for evaluation. A better approach is to allow a single object
%to be created that supports both the printing and evaluation features.
%Unfortunately, dynamic composition of algebras is non-trivial.
\citet{oliveira2012extensibility} addressed this problem by proposing
\textit{Object Algebra combinators} that combine multiple algebras into one.
However, as they noted, such combinators written in Java are difficult to use in
practice, and they require significant amounts of boilerplate. Improved variants
of Object Algebra combinators have been encoded in Scala using intersection
types and an encoding of the merge construct~\cite{oliveira2013feature,
  rendel14attributes}. Unfortunately, the Scala encoding of the merge construct
is quite complex as it relies on low-level type-unsafe programming features such
as dynamic proxies, reflection or other meta-programming techniques. In \name
there is no need for such complex encoding, as the merge construct is natively
supported. This allows type-safe, coherent and boilerplate-free composition of
Object Algebras.

% \paragraph{Merge in \name} Section~\ref{sec:scala-merge} points out that in Scala
% intersection types are not completely first class, since it is hard to
% build values of intersections with unknown components. In \name,
% however, this is trivial. The \lstinline{merge} operation discussed
% in Section~\ref{sec:scala-merge} can be written as:
% \lstinputlisting[linerange=4-4]{../examples/merge.txt}% APPLY:linerange=MERGE_EG

% \noindent From the typing point of view, the difference between
% \lstinline{merge} in Scala and \name is that the type variable
% \lstinline{B} now has a \emph{disjointness constraint}. The notation
% \lstinline{B * A} means that the type variable \lstinline{B} can be instantiated
% to any types that are disjoint to the type \lstinline{A}. This is
% crutial to ensure that the composition between \lstinline{x} and \lstinline{y}
% is well-formed. That is, there are no conflicts between the two objects.

\paragraph{Boilerplate-free composition of Object Algebras}
%An acute reader may have smelled some boilerplate code in the definition of
%\lstinline{combine}. After all, \lstinline{combine} just invokes the
%corresponding method on the arguments, which is solely driven by the types of
%the two input algebras and the resulting algebra.
The merge construct (cf. Section~\ref{sec:merge-construct}) in \name is quite
powerful. Using trait inheritance (which is just syntactic sugar to the merge
construct) it is possible to compose two Object Algebras in a trivial way:

%is, in general, the type
%system behind \name is powerful enough to derive such definition for us:
%\lstinline{combine} is no more than just the merge of two algebras!

\lstinputlisting[linerange=57-57]{../examples/visitor3.txt}% APPLY:linerange=COMBINE_DEF

That is it. None of the boilerplate in other approaches~\cite{oliveira2012extensibility}, or type-unsafe
meta-programming techniques~\cite{oliveira2013feature,rendel14attributes} of other approaches are needed! Three points are
worth noting: 1) \lstinline{combine} relies on \textit{dynamic inheritance}.
Notice how we let \lstinline{combine} inherit \lstinline{f & g}, for which the
implementation is unknown statically; 2) \name supports multiple inheritance of
traits with the same type: both \lstinline{f} and \lstinline{g} are traits of
the \emph{same} subtraction algebra \lstinline{SubExpAlg} and 3) disjointness
constraint (\lstinline{B * A}) is \textit{crucial} to make sure two Object
Algebras (\lstinline{f} and \lstinline{g}) are disjoint in order to avoid
conflicts when being composed. Type systems for many OO languages are too
inflexible to express this kind of flexible and dynamic inheritance. \name hits
a sweet spot in the design space of being dynamic, while retaining much
of the flexibility of dynamic languages, and without sacrificing the type-safety and
coherence guarantees.

With \lstinline{combine} at hand, we can finally have an object \lstinline{o}
that is able to print and evaluate at the same time:

\lstinputlisting[linerange=66-69]{../examples/visitor3.txt}% APPLY:linerange=COMBINE1_TEST


\begin{comment}
which is of type
\lstinline{Exp}, to \lstinline{sub}, which expects a value of type
\lstinline{ExtExp}. This works precisely because \lstinline{Exp} is a
\textit{subtype} of \lstinline{ExtExp}. An acute reader by now may find an issue
about code reuse, i.e., we cannot reuse constructors! For example, the following
is rejected:
\lstinputlisting[linerange=75-75]{../examples/visitor.txt}% APPLY:linerange=EXPRESSION_WRONG
Instead, \lstinline{e3} should be written as follow:
\lstinputlisting[linerange=-]{}% APPLY:linerange=EXPRESSION_CORRECT
Although, admittedly, creating expressions like this is slightly more
cumbersome.
\end{comment}


\begin{comment}
The combinator is defined by the \lstinline{combine} trait, which takes two
algebras to create a combined algebra. It does so by appropriately delegating
behaviours in each component algebra to the combined algebra.

\lstinputlisting[linerange=-]{}% APPLY:linerange=COMBINE1_DEF

\bruno{Well, the previous subsection already introduces parametric
  polymorphism, so this discussion comes too late!}
Something new appears the above trait declaration. \name supports parametric
polymorphism. In \name, type variables use uppercase letters. Following the work
of~\citet{alpuimdisjoint}, \name uses an extension to universal quantification
called \textit{disjoint quantification}, where a type variable can be
constrained so that it is disjoint with a given type (\lstinline{B * A} for
example). Parametric polymorphism is need because \lstinline{combine} must
compose algebras with arbitrary type parameters. A disjointness constraint is
needed to ensure that two input algebras build values of disjoint types
(otherwise ambiguity could arise).
\end{comment}

\begin{comment}
\paragraph{Data constructors.} Using \lstinline{Exp} the two data constructors
are defined as follows:
\lstinputlisting[linerange=16-21]{../examples/visitor.txt}% APPLY:linerange=DC_DEF
% Note that the notation \lstinline{/\E} is type abstraction: it introduces a type
% variable in the environment.
\end{comment}

\begin{comment}
In a
latter paper, Oliveira et al.~\cite{} noted some limitations of the
original design pattern and proposed some new techniques
that generalized the original pattern, allowing it to express
programs in a Feature-Oriented Programming~\cite{} style. Key
to these techniques was the ability to dynamically compose
Object Algebras.

Unfortunatelly, dynamic composition of Object Algebras
is non-trivial. At the type-level it is possible to express the
resulting type of the composition using intersection types.
Thus, it is still possible to solve that part problem nicely in a
language like Scala (which has basic support for intersection
types). However, the dynamic composition itself cannot be
easily encoded in Scala. The fundamental issue is that Scala
lacks a merge operator (see the discussion in Section~\ref{}).
Although both Oliveira et al.~\cite{} and Rendell et al.~\cite{}
have shown that such a merge operator can be encoded in
Scala, the encoding fundamentally relies in low-level programming
techniques such as dynamic proxies, reflection or
meta-programming.

Because \name supports a merge operator natively, dynamic
Object Algebra composition becomes easy to encode. The
remainder of this section shows how Object Algebras and
Object Algebra composition can be encoded in \name. We will
illustrate this point step-by-step by solving the Expression
Problem~\cite{}.

\name provides two main advantages:

\begin{itemize}

\item {\bf First-class Object Algebra values:} Due to the separation
  of inheritance and subtyping, \name can express first-class Object
  Algebra values. The key aspect is that AST extensions become
  supertypes of the original ASTs. This is unexpressible in many
  languages, but expressible in \name.

\item {\bf Built-in composition for Object Algebras:} The inheritance
  mechanisms of \name can express Object Algebra delegation-based composition operators~\cite{}
  directly. This is in contrast to previous work, where such operators
  have to be encoded, with significant amounts of boilerplate, with
  either pairs or intersection types. Moreover, the built-in support
  for composition operators \emph{guarantees} safe and unambigous
  composition, which is not guaranteed in previous work.

\end{itemize}

Various solutions to the Expression Problem~\cite{wadler1998expression} in the
literature~\cite{finally-tagless,oliveira09modular,DelawareOS13,oliveira2012extensibility,
  swierstra:la-carte} are closely related to type-theoretic encodings of
datatypes. Indeed, variants of the same idea keep appearing in different
programming languages, because the encoding of the idea needs to exploit the
particular features of the programming language (or theorem prover).
Unfortunately language-specific constructs obscure the key ideas behind those
solutions. In this section we presents \name's solution to the Expression
Problem that intends to capture the key ideas of various solutions in the
literature.

\subsection{Extensibility and Subtyping}
\label{sec:extensibility}
\end{comment}