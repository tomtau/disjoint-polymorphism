
\section{First-class Object Algebra Values: A Case for ``Inheritance is not
  Subtyping''}

Various solutions to the Expression Problem~\cite{wadler1998expression} in the
literature~\cite{finally-tagless,oliveira09modular,DelawareOS13,oliveira2012extensibility,
  swierstra:la-carte} are closely related to type-theoretic encodings of
datatypes. Indeed, variants of the same idea keep appearing in different
programming languages, because the encoding of the idea needs to exploit the
particular features of the programming language (or theorem prover).
Unfortunately language-specific constructs obscure the key ideas behind those
solutions. In this section we presents \name's solution to the Expression
Problem that intends to capture the key ideas of various solutions in the
literature.

\subsection{Extensibility and Subtyping}
\label{sec:extensibility}

To start, we begin with a very simple system modeling arithmetic expressions and
evaluation. The initial system constitutes expressions with two variants
(literals and addition), as depicted below:
\lstinputlisting[linerange=4-7]{../examples/visitor.txt}% APPLY:linerange=ALGEBRA_DEF

The first part, captured by the type \lstinline{ExpAlg[E]}, constitutes the
so-called algebra of the datatype. The first component abstracts over the type
of literal expressions (\lstinline{Int -> E}). The second component abstracts
over the type of addition expression (\lstinline{E -> E -> E}).

The second part, which is the actual type of the arithmetic expression, is:
\lstinputlisting[linerange=11-11]{../examples/visitor.txt}% APPLY:linerange=CHURCH_DEF
An attentive reader may immediately recognize this is exactly the VISITOR
pattern: the method \lstinline{accept} takes an internal visitor as argument.

% \paragraph{Data constructors.} Using \lstinline{Exp} the two data constrcutors
% are defined as follows:
% \lstinputlisting[linerange=52-61]{../examples/visitor.txt}% APPLY:linerange=DC_DEF
% Note that the notation \lstinline{/\E} is type abstraction: it introduces a type
% variable in the environment.


\paragraph{Operations.} Defining operations over expressions requires
implementing \lstinline{ExpAlg[E]}. An interesting operation over expressions is
evaluation. The first step is to choose a suitable concrete type for
instantiating the type parameter \lstinline{E}. One such suitable type is:
\lstinputlisting[linerange=15-15]{../examples/visitor.txt}% APPLY:linerange=EVAL_DEF
Using \lstinline{IEval}, we can define a trait that implements the evaluation
rule for each variant:
\lstinputlisting[linerange=19-22]{../examples/visitor.txt}% APPLY:linerange=EVAL_IMPL


\paragraph{Add a subtraction variant.} Interesting things happen when a new
variant, such as subtraction is added. To do so, we need to extend both
\lstinline{ExpAlg[E]} and \lstinline{evalAlg}:
\lstinputlisting[linerange=27-31]{../examples/visitor.txt}% APPLY:linerange=SUB_DEF
Firstly, \lstinline{SubExpAlg[E]} defines an extended algebra that contains the
variants of the original plus the new subtraction variant. Intersection types
are used to model \textit{type inheritance}. Secondly, a new type of expressions
(\lstinline{ExtExp}) with subtraction is needed. It is important to note that
the \lstinline{accept} method now takes the new algebra \lstinline{SubExpAlg[E]}
as argument.

\paragraph{Inheritance is not subtyping.} In the presence of subtyping, there
are interesting subtyping relations between datatypes and their
extensions~\cite{oliveira09modular}. An interesting aspect about subtyping in
the new system is that subtyping follows the opposite direction of the
extension. In other words, subtyping is contravariant with respect to the
extension. First note that \lstinline{SubExpAlg[E]} appears in parameter
position of \lstinline{accept} and function parameters are naturally
contravariant: \lstinline{ExtExp} should be a \textit{supertype} of
\lstinline{Exp}. By contrast, in common OO languages, type inheritance and
subtyping always go along together. For example, Scala forbids any kind of
type-refinement on method parameter types. The consequence of this is that in
those languages, it is impossible to express that \lstinline{ExtExp} is both an
extension and a supertype of \lstinline{Exp}. Such kind of phenomena is an
example where ``inheritance is not subtyping''~\cite{cook1989inheritance}.

\paragraph{Add a new operation.} The second type of extension is adding a new
operation, such as pretty printing. Similar to evaluation, the interface of the
pretty printing feature is modeled as:
\lstinputlisting[linerange=37-37]{../examples/visitor.txt}% APPLY:linerange=PRINT_DEF
The implementation is spelled out straightforwardly:
\lstinputlisting[linerange=42-46]{../examples/visitor.txt}% APPLY:linerange=PRINT_IMPL


\paragraph{Usage.} The last piece is to define convenient data constructors:
\lstinputlisting[linerange=52-61]{../examples/visitor.txt}% APPLY:linerange=DC_DEF
So now we can test the shiny new system:
\lstinputlisting[linerange=66-70]{../examples/visitor.txt}% APPLY:linerange=VISITOR_EG
Notice that we are able to pass \lstinline{lit 5}, which is of type
\lstinline{Exp}, to \lstinline{sub}, which expects a value of type
\lstinline{ExtExp}. This works precisely because \lstinline{Exp} is a
\textit{subtype} of \lstinline{ExtExp}.



\subsection{How much dynamic is too much}
\label{sec:dynamic}

The solution to the Expression Problem presented in
Section~\ref{sec:extensibility} has one wrinkle in its usage. To be able to call
different operations from the same expression \lstinline{e}, we need to make two
copies \lstinline{o1} and \lstinline{o2}: one is used for printing, while the
other is used for evaluation. An ideal approach would be allowing a single object
to be created that supports both the printing and evaluation features.
Unfortunately, dynamic composition of algebras is non-trivial.
\citet{oliveira2012extensibility} addressed this problem by proposing
\textit{object algebra combinators} that combine multiple algebras into one.
However, as they noted, such combinators written in Java are difficult to use in
practice. A version of this combinator has been encoded in Scala using
intersection types and an encoding of the merge
operator~\cite{oliveira2013feature, rendel14attributes}. Again, the Scala
encoding of the merge operator is quite complex as it relies on low-level
type-unsafe programming features such as dynamic proxies, reflection or other
meta-programming techniques. In \name there is no need for such complex
encoding, as the merge operator is natively supported.

The combinator is defined by the \lstinline{combine} trait, which takes two
algebras to create a combined algebra. It does so by appropriately delegating
behaviours in each component algebra to the combined algebra.

\lstinputlisting[linerange=49-53]{../examples/visitor3.txt}% APPLY:linerange=COMBINE1_DEF

Something new appears the above trait declaration. \name supports parametric
polymorphism. In \name, type variables use uppercase letters. Following the work
of~\citet{alpuimdisjoint}, \name uses an extension to universal quantification
called \textit{disjoint quantification}, where a type variable can be
constrained so that it is disjoint with a given type (\lstinline{B * A} for
example). Parametric polymorphism is need because \lstinline{combine} must
compose algebras with arbitrary type parameters. A disjointness constraint is
needed to ensure that two input algebras build values of disjoint types
(otherwise ambiguity could arise).

With \lstinline{combine} at hand, we can finally have a object \lstinline{o}
print and evaluate at the same time:

\lstinputlisting[linerange=71-74]{../examples/visitor3.txt}% APPLY:linerange=COMBINE1_TEST

An acute reader may have smelled some boilerplate code in the definition of
\lstinline{combine}. After all, \lstinline{combine} just invokes the
corresponding method on the arguments, which is solely driven by the types of
the two input algebras and the resulting algebra. It turns out that the type
system behind \name is powerful enough to derive such definition for us:
\lstinline{combine} is no more than just the merge of two algebras!

\lstinputlisting[linerange=59-59]{../examples/visitor3.txt}% APPLY:linerange=COMBINE_DEF

That's it. No more boilerplate necessary. Two points worth noting: 1) \name
supports \textit{dynamic inheritance}. Notice how we let \lstinline{combine1}
``inherits'' \lstinline{f & g}, for which the implementation is unknown
statically. 2) \name supports multiple inheritance of traits with the same type:
both \lstinline{f} and \lstinline{g} are traits of the subtraction algebra. Type
systems for many OO languages are too inflexible to express this kind of dynamic
inheritance. \name hits a sweet spot in the design space of being how much
dynamic, where we retain as much flexibility as in a dynamic language without
sacrificing the type-safety guarantee we always love from a static type system.
