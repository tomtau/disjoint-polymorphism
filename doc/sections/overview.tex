\section{A Critique of Mainstream Statically-Typed OOP}
\label{sec:critique}

This section discusses several limitations of mainstream OOP languages
that work against modularity and reuse. The limitations discussed here
are relevant in a concrete practical application: Extensible Visitors and/or
Object Algebras. Such limitations arise from the \emph{covariant
  model} of objects, the limitations imposed by the use of
\emph{static inheritance} and common (overly conservative)
\emph{restrictions} imposed by inheritance mechanisms.  Our discussion
serves as a critique to the mechanisms present in current mainstream
statically-typed OOP languages, and it is a motivator for the design
of \name. Section~\ref{sec:OA} shows how Object Algebras and Extensible
Visitors are modelled in \name.  However, independently of our work on
\name, the critique is useful to pinpoint drawbacks that could be
addressed differently in future work by others.

\begin{comment}
 In particular we argue that in the
standard model of class-based OOP is grounded on the following 
assumptions:

\begin{itemize}

\item {\bf The covariant view of objects:} 

\item {\bf The view that extensions lead to subtypes:}

\item {\bf }

\end{itemize}

Those points were a driver for the
design of \name, which aims to address all of those limitations. 

\subsection{The Standard Model of Class-Based OOP}

In the standard statically typed OOP model for class-based languages
\emph{extensions} (subclasses or subinterfaces) are viewed as \emph{subtypes} of the original
class/interface. Moreover subclasses inherit from the superclasses,
and inheritance and subtyping always go along together.
This is the model adopted by most popular mainstream
languages, such as Java, C\#, C++ or Scala. This model can be traced
back to the origins of OOP in Simula~\cite{}, and has shown an 
undeniable resilience through times. One reason why this model remains
so popular is probably simplicity: 
extensions (of classes or interfaces) always result in subtypes.  
This results in a very simple programming model, which is easy 
to explain to programmers.  

Unfortunately, as the theory of OOP languages and subtyping has taught
us that the story is not quite so simple. Since the early days of the
theoretical foundations of OOP and subtyping, it has bee

Early on Cook et al. argued
that {\bf Inheritance is not Subtyping}. \ldots
\end{comment}

\subsection{No Retroactive Super Types and Contravariant Subtyping Refinement}

In the covariant model extensions are always subtypes. However, 
there are situations where extensions are actually
\emph{supertypes} (or can even have no relation to the extended type). Unfortunately this is not supported in the
covariant model. Using some imaginary syntax for a Java-like language, 
what we would like to express is:

\begin{lstlisting}
interface B {}   // original interface
interface A super B {} // Interface A added later, which becomes a supertype of B
\end{lstlisting}

\noindent We call such an imaginary feature \emph{retroactive super types}. The
feature is said to be retroactive because supertypes can be added \`a
posteriori, which is not possible in languages like Java. We should note that
\citet{ostermann2008nominal} also proposes a similar feature to retroactive
super types.

Retroactive super types could also work with classes, and
inheritance. In this case it would be important to break the
functionality that is currently provided by \lstinline[language=java]{extends}, so that
code can be inherited without creating a subtype at the same time. 
A similar example with classes, but also accounting for inheritance 
would be:

\begin{lstlisting}
class B {}   // original class
//class A is added later, inherits from B and becomes a supertype of B
class A inherits B super B {}
\end{lstlisting}

\noindent Once the functionality of \lstinline[language=java]{extends} is broken, it also
becomes possible to inherit code without creating any subtype
relation. This is for example the situation that is discussed by \citet{cook1989inheritance}.

\paragraph{Contravariant Type Refinement}
Another feature that is missing from mainstream OOP languages is 
contravariant refinement of argument types. Most Java-like OOP languages support covariant return type 
refinement:

\begin{lstlisting}[language=java]
interface A  {A m();} 
interface B  extends A {B m();} // B extends and refines the return type of m() to a subtype
\end{lstlisting}

\noindent From the theory of OOP languages, we know that an equally valid form
of refinement also applies to method arguments~\cite{cardelli1984semantics}.
However the refinement on method arguments is \emph{contravariant}. For an extended
type to become a subtype under argument refinements in methods, it is necessary
that the arguments of the methods in the subtypes become supertypes. The
following example illustrates this:

\begin{lstlisting}[language=java]
interface A  {Int m(String s);} 
interface B  extends A {Int m(Object s);} // Invalid: B extends and refines the argument type of m()
\end{lstlisting}

\noindent Here the intention is to refine the argument of method
\lstinline{m} with a supertype (\lstinline{Object}) so that the
interface \lstinline{B} becomes a subtype. Unfortunately this is invalid in Java, since Java
does not support contravariant type refinement of arguments. 
A possible reason for not having such feature is that there are not
many applications where this feature \emph{in isolation} seems necessary. 

\begin{figure}
\begin{lstlisting}[language=java]
interface ExpVisitor<A> { // Base Visitor and Language
	A lit(int x);
	A add(A e1, A e2);
}
interface Exp { // Simple AST for expressions
	<A> A accept(ExpVisitor<A> v);
}
interface ExtExpVisitor<A> extends ExpVisitor<A> { // Extended Visitor and Language
	A sub(A e1, A e2);
}
interface ExtExp super Exp { // An extended AST becomes a supertype of the original one
	<A> A accept(ExtExpVisitor<A> v); // Contravariant argument refinement
}
\end{lstlisting}
\caption{Extensible Visitors in an hypothetical
  extension of Java with \emph{retroactive super types} and \emph{contravariant
  type refinement}.}
\label{fig:JVis}
\end{figure}

\paragraph{Combining Retroactive Super Types with Contravariant Type Refinement}

However, combined with retroactive super types, contravariant refinement of
argument types becomes much more interesting! In particular we claim that this
combination is helpful in dealing with modularity/reuse problems that arise in
Object-Algebras/Visitors. There have been several attempts at making visitors
extensible in the past~\cite{oliveira09modular,togersen:2004}. However most
attempts usually rely on advanced type system features. With retroactive super
types and contravariant refinement, extensible visitors and improved variants of
Object Algebras could be achieved more naturally.

Figure~\ref{fig:JVis} presents pseudo Java code that uses the two features in
combination. The interface \lstinline{ExpVisitor} represents a (internal)
visitor interface (and also an Object Algebra
interface~\cite{oliveira2012extensibility}) for a simple AST containing literals
and additions. The interface \lstinline{Exp} represents the AST type. In a
later extension, a subtraction node is added. Modelling the extended visitor
interface (\lstinline{ExtExpVisitor}) is easy: just add the new visit method
(\lstinline{sub}). The interesting part is modelling the extended AST type. The
extended type (\lstinline{ExtExp}) should become a \emph{supertype} of
\lstinline{Exp}. This follows from the fact that the new \lstinline{accept}
method takes an argument which is a subtype of the argument in the original \lstinline{accept}
method. Thus we get the following subtyping relation between the signatures of
the two \lstinline{accept} methods:

\begin{lstlisting}[mathescape=true]
ExpVisitor<A> $\to$ A $<:$ ExtExpVisitor<A> $\to$ A
\end{lstlisting}

\noindent Here the subtype (the type on the left) is the type of the
\lstinline{accept} method in \lstinline{Exp}, and the supertype is the type of
the new \lstinline{accept} method. Since the AST type is just the
\lstinline{accept} method, it is clear that the extended AST type should be a
supertype of the original one. 

Although such code cannot be expressed in Java, it is expressible in
\name, as shown in Section~\ref{sec:OA}. In fact statically-typed
OO languages using structural types (and which also support
contravariant type refinement!) should be able to express 
the example, as the subtyping relations are implicit.

\subsection{An Overly Conservative Restriction on Multiple Inheritance}

Multiple inheritance can be quite helpful to provide reuse. Although not all
mainstream statically-typed OOP languages support multiple inheritance, some of
them do. For example Scala, C++ and Java 8 (with default methods) all support
some form of multiple inheritance. However, there are some restrictions to the
usage of multiple inheritance. One of those common restrictions is that the same
class (or trait) cannot be inherited twice. We illustrate this restriction using
Scala traits (which provide a form of multiple implementation inheritance):

\begin{lstlisting}[language=scala]
trait A { def m(x : A) = x }
trait B extends A with A {}
\end{lstlisting}

\noindent The code is rejected with the message: ``trait A is
inherited twice''. At first sight such restriction appears
reasonable. However consider the following variant of the example 
above:

\begin{lstlisting}[language=scala]
trait C[A] { def m(x : A) = x }
trait D extends C[Int] with C[Boolean]
\end{lstlisting}

\noindent Here we create a \emph{parameterized} trait \lstinline{C},
defining a method \lstinline{m}. Then the trait \lstinline{D} attempts 
to inherit from \lstinline{C} twice: the first time with the type
parameter instantiated to \lstinline{Int}, and the second time with
the type parameter instantiated to \lstinline{Boolean}. The intention 
would be that the resulting class would have two \emph{overloaded}
methods \lstinline{m}. Unfortunately this code is rejected for the
same reason as the previous example: the trait \lstinline{C} is
inherited twice. To overcome this problem, a possible solution would
be to create two specialized versions of \lstinline{C}:

\begin{lstlisting}[language=scala]
trait CInt { def m(x : Int) = x }
trait CBoolean { def m(x : Boolean) = x }
trait D extends CInt with CBoolean
\end{lstlisting}

\noindent which works, but it is non-modular, as it requires changing 
the original definition of \lstinline{C}. 

In general, while for non-parametric classes/traits the restriction
that the same class cannot be inherited twice makes sense, for parametric
classes there are situations where it would make sense to inherit 
from the same class/trait twice. As it turns out similar situations 
do arise in practice when defining composition operators for
Object Algebras, as we shall see in Section~\ref{sec:dynamic}.

\subsection{No Dynamic Inheritance}
\label{sec:scala-merge}

The final limitation is that mainstream OOP languages use a class-based static
inheritance model. This means that when both inheritance is used and objects are
created, the \emph{classes involved must be statically known}.
\begin{comment}
Intersection types date back as early as Coppo et al.'s
work~\cite{coppo1981functional}. Since then researchers have studied
intersection types, and some languages have adopted them in one form or another.

In Java, for example,
\begin{lstlisting}[language=java]
interface AwithB extends A, B {}
\end{lstlisting}
introduces a new interface \lstinline$AwithB$ that satisfies the interface of
both \lstinline{A} and \lstinline{B}. 
\end{comment}
In Scala, given two concrete traits, it is
possible to use \textit{mixin composition} to create an object that implements
both traits.
\begin{lstlisting}[language=scala]
trait A
trait B
val newAB : A with B = new A with B
\end{lstlisting}
\noindent Here the type \lstinline[language=scala]{A with B} is an intersection type,
and the expression \lstinline[language=scala]{new A with B} allows creating a new
object from the combination of the traits \lstinline{A} and
\lstinline{B}.

Scala also allows intersections of type parameters. For example,
\begin{lstlisting}[language=scala]
def merge[A,B] (x: A) (y: B) : A with B = ...
\end{lstlisting}
uses the anonymous intersection of two type parameters \lstinline{A} and
\lstinline{B}.

However, in Scala it is not possible to dynamically compose two
(statically unknown) \emph{objects}. For
example, the following code:
\begin{lstlisting}[language=scala]
// Invalid Scala code:
def merge[A,B] (x: A) (y: B) : A with B = new x with y
\end{lstlisting}
is rejected by the Scala compiler. The problem is that the \lstinline[language=scala]$with$
construct in Scala can only be used to mixin traits or classes, not
arbitrary objects. This limitation essentially puts intersection types in Scala
in a second-class status. Although \lstinline{merge} returns an intersection
type, it is hard to build values with such types.

Exactly the same situation arises when using inheritance. In Scala, if we try to
create a trait parameterized by an unknown supertype, for example
\lstinline[language=scala]{trait T[A] extends A}, we face problems. Here the
intention would be to capture a family of traits that extend
multiple different traits/classes with the same method. However, this definition
is rejected in Scala because \lstinline[language=scala]{extends} can only be
used with concrete classes (and not type parameters).
 
%A possible solution is to use
%low-level meta-programming mechanisms (such as run-time
%reflection)~\cite{}, but clearly it would be better to have . 
%However, this relies on
%type-unsafe operations, and can result in run-time type exceptions
%that should be caught statically instead.

The need for dynamic inheritance arises in composition operators for Object
Algebras. Although such compositions can be encoded in Scala using low-level
meta-programming mechanisms, a language-based solution would be clearly better.
Section \ref{sec:dynamic} shows that \name offers such language-based solution.
