\section{Applications}

% Report on various applications encoded with \name. Conduct a larger case study with visitors.

This section shows the expressiveness of \name. We show that features of \name
(dynamically composable traits, intersection types, the merge operator,
parametric polymorphism and disjoint quantification) enable extensible designs
that have been presented in mainstream languages. In particular, \name address
limitations of those languages, making the designs significantly simpler. In
what follows, we preset a case study modularizing several orthogonal features of
a variant of JavaScript, called mini-JS. The case study illustrates that how
various features and operations can be modularly developed and later composed to
assembly a complete language with various operations baked in.

\subsection{The Base Language}

We start with a very simple language of arithmetic expressions.
\lstinputlisting[linerange=111-120]{../examples/visitor2.txt}% APPLY:linerange=BASE_DEF

The base language has features such as literals, addition, subtraction among
others. As with examples in Section~\ref{sec:extensibility}, we use visitors to
encode datatypes. Next we define three useful operations on top of this simple
language.

\subsubsection{Evaluator}

The evaluation interface is shown below:
\lstinputlisting[linerange=47-55]{../examples/visitor2.txt}% APPLY:linerange=EVALUATION
Two points worth mentioning: 1) Boolean values (\lstinline{boolV}) in the
\lstinline{Value} datatype is not needed yet, but it will be useful later when
we extend the base language; 2) the \lstinline{eval} method is a function from
an evaluation environment (\lstinline{Env}) to \lstinline{Maybe[Value]}, again
at this stage, the evaluation environment is not used, but will be useful when
we have variables.

Here is the full code of the evaluator.
\lstinputlisting[linerange=126-148]{../examples/visitor2.txt}% APPLY:linerange=BASE_EVALUATOR
To reduce boilerplate code when dealing with \lstinline{Maybe} values, like in
Haskell, we encode the bind operator \lstinline{bind} of the \lstinline{Maybe}
monad. The code is rather boring, with \lstinline{numV} and \lstinline{fromNum}
as constructor and destructor of the \lstinline{Value} type.

\subsubsection{Pretty printer}

The pretty printer is constructed in a similar manner.
\lstinputlisting[linerange=75-75]{../examples/visitor2.txt}% APPLY:linerange=PRETTY_PRINTER
\lstinputlisting[linerange=153-159]{../examples/visitor2.txt}% APPLY:linerange=PRETTY_BASE

\subsubsection{Type checker}

The type checker interface is shown below:
\lstinputlisting[linerange=85-93]{../examples/visitor2.txt}% APPLY:linerange=TYPECHEKER
Like in the evalutor, Boolean type (\lstinline{tbool}) and the typing
environment (\lstinline{TEnv}) are not used for now.

Here is the full code of the type checker:
\lstinputlisting[linerange=164-202]{../examples/visitor2.txt}% APPLY:linerange=TYPECHECKER_BASE
It looks quite similar with the evaluator. \lstinline{caseType} is like doing
case analysis of the \lstinline{Type} values. Again the code is unremarkable, as
would be done in a functional language like Haskell.


\subsection{Adding Conditional Expressions}

Now we extend the base language with conditional expressions.
\lstinputlisting[linerange=216-223]{../examples/visitor2.txt}% APPLY:linerange=CONDITIONAL
The augmented language now has comparison features (\lstinline{eq} and
\lstinline{le}) and if expressions. Again we emphasize that
\lstinline{BoolArith} inherits features of \lstinline{Arith}, and is a
\textit{supertype} of \lstinline{Arith}.

Here is the extended evaluator. Now we need Boolean values.
\lstinputlisting[linerange=228-245]{../examples/visitor2.txt}% APPLY:linerange=EVALUATOR_CONDITIONAL
Note that we didn't touch anything from \lstinline{evalArithAlg} but extend it
to from a new evaluator. Now because the language can produce two kinds of
values, it is worth noting that \lstinline{fromNum} (or \lstinline{fromBool}) is
partial, meaning that it can diverge at run-time if the value is not a number
(or Boolean). This is why we need a type checker to make sure such things cannot
happen at run-time.

Same goes for the pretty printer and the type checker.
\lstinputlisting[linerange=250-255]{../examples/visitor2.txt}% APPLY:linerange=PRETTY_CONDITIONAL
\lstinputlisting[linerange=261-294]{../examples/visitor2.txt}% APPLY:linerange=TYPECHECKER_CONDITIONAL
\jeremy{too long and boring, can we omit it}



\subsection{Adding Variable Declarations}

Next we extend our language again with local variables and variable
declarations. We add two new constructors (\lstinline{var} and
\lstinline{decl}).
\lstinputlisting[linerange=308-314]{../examples/visitor2.txt}% APPLY:linerange=VARIABLE
For simplicity, we encode variables using strings. \lstinline{decl} takes three
arguments, the first is the variable to be declared, the second is the
expression bound with the variable, and the third is the body within which the
declared variable is in scope. Now the language is interesting enough to express
something like \lstinline{var x = 3; x + 1}.

The evaluator finally makes use of the evaluation environment, which is a
mapping from variables to values.
\lstinputlisting[linerange=319-325]{../examples/visitor2.txt}% APPLY:linerange=EVALUATOR_VARIABLE
For the variable case, we look it up in the environment to find its bound value.
For the declaration case, we evaluate the body expression in the augmented
environment.

Here are the pretty printer and the type checker.
\lstinputlisting[linerange=330-333]{../examples/visitor2.txt}% APPLY:linerange=PRETTY_VARIABLE
\lstinputlisting[linerange=338-344]{../examples/visitor2.txt}% APPLY:linerange=TYPECHECKER_VARIABLE


\subsection{Putting All Together}

We hope by now the reader can share our feeling that this is indeed a modular
way to cook a language incrementally. We can of course continue in the same
fashion with more features and operations. We now stop to use the language to
make sure everything works as we expected.

First let us create an expression that uses all the features the language
supports now.
\lstinputlisting[linerange=349-357]{../examples/visitor2.txt}% APPLY:linerange=FEATURES
The concrete syntax of the expression is shown in the comment above.

Now we can of course apply the operations to the expression individually.
Instead of doing this, we go a step further -- we merge those three operations
into one, and apply the combined operation to the expression.
\lstinputlisting[linerange=412-414]{../examples/visitor2.txt}% APPLY:linerange=COMBINED
\lstinline{superAlgInst} is the combined operation that has the ability to type
check, evaluate, and print a given expression.

Here is a function that makes sure ``well-typed programs cannot go wrong'':
\lstinputlisting[linerange=418-422]{../examples/visitor2.txt}% APPLY:linerange=SUPER_FN

We use it on the expression as follows:
\lstinputlisting[linerange=426-427]{../examples/visitor2.txt}% APPLY:linerange=VARIABLE_TEST
Everything works as expected!